% Options for packages loaded elsewhere
\PassOptionsToPackage{unicode}{hyperref}
\PassOptionsToPackage{hyphens}{url}
\PassOptionsToPackage{dvipsnames,svgnames,x11names}{xcolor}
%
\documentclass[
  letterpaper,
  DIV=11,
  numbers=noendperiod,
  oneside]{scrreprt}

\usepackage{amsmath,amssymb}
\usepackage{lmodern}
\usepackage{iftex}
\ifPDFTeX
  \usepackage[T1]{fontenc}
  \usepackage[utf8]{inputenc}
  \usepackage{textcomp} % provide euro and other symbols
\else % if luatex or xetex
  \usepackage{unicode-math}
  \defaultfontfeatures{Scale=MatchLowercase}
  \defaultfontfeatures[\rmfamily]{Ligatures=TeX,Scale=1}
\fi
% Use upquote if available, for straight quotes in verbatim environments
\IfFileExists{upquote.sty}{\usepackage{upquote}}{}
\IfFileExists{microtype.sty}{% use microtype if available
  \usepackage[]{microtype}
  \UseMicrotypeSet[protrusion]{basicmath} % disable protrusion for tt fonts
}{}
\makeatletter
\@ifundefined{KOMAClassName}{% if non-KOMA class
  \IfFileExists{parskip.sty}{%
    \usepackage{parskip}
  }{% else
    \setlength{\parindent}{0pt}
    \setlength{\parskip}{6pt plus 2pt minus 1pt}}
}{% if KOMA class
  \KOMAoptions{parskip=half}}
\makeatother
\usepackage{xcolor}
\usepackage[left=1in,marginparwidth=2.0666666666667in,textwidth=4.1333333333333in,marginparsep=0.3in]{geometry}
\setlength{\emergencystretch}{3em} % prevent overfull lines
\setcounter{secnumdepth}{5}
% Make \paragraph and \subparagraph free-standing
\ifx\paragraph\undefined\else
  \let\oldparagraph\paragraph
  \renewcommand{\paragraph}[1]{\oldparagraph{#1}\mbox{}}
\fi
\ifx\subparagraph\undefined\else
  \let\oldsubparagraph\subparagraph
  \renewcommand{\subparagraph}[1]{\oldsubparagraph{#1}\mbox{}}
\fi


\providecommand{\tightlist}{%
  \setlength{\itemsep}{0pt}\setlength{\parskip}{0pt}}\usepackage{longtable,booktabs,array}
\usepackage{calc} % for calculating minipage widths
% Correct order of tables after \paragraph or \subparagraph
\usepackage{etoolbox}
\makeatletter
\patchcmd\longtable{\par}{\if@noskipsec\mbox{}\fi\par}{}{}
\makeatother
% Allow footnotes in longtable head/foot
\IfFileExists{footnotehyper.sty}{\usepackage{footnotehyper}}{\usepackage{footnote}}
\makesavenoteenv{longtable}
\usepackage{graphicx}
\makeatletter
\def\maxwidth{\ifdim\Gin@nat@width>\linewidth\linewidth\else\Gin@nat@width\fi}
\def\maxheight{\ifdim\Gin@nat@height>\textheight\textheight\else\Gin@nat@height\fi}
\makeatother
% Scale images if necessary, so that they will not overflow the page
% margins by default, and it is still possible to overwrite the defaults
% using explicit options in \includegraphics[width, height, ...]{}
\setkeys{Gin}{width=\maxwidth,height=\maxheight,keepaspectratio}
% Set default figure placement to htbp
\makeatletter
\def\fps@figure{htbp}
\makeatother
\newlength{\cslhangindent}
\setlength{\cslhangindent}{1.5em}
\newlength{\csllabelwidth}
\setlength{\csllabelwidth}{3em}
\newlength{\cslentryspacingunit} % times entry-spacing
\setlength{\cslentryspacingunit}{\parskip}
\newenvironment{CSLReferences}[2] % #1 hanging-ident, #2 entry spacing
 {% don't indent paragraphs
  \setlength{\parindent}{0pt}
  % turn on hanging indent if param 1 is 1
  \ifodd #1
  \let\oldpar\par
  \def\par{\hangindent=\cslhangindent\oldpar}
  \fi
  % set entry spacing
  \setlength{\parskip}{#2\cslentryspacingunit}
 }%
 {}
\usepackage{calc}
\newcommand{\CSLBlock}[1]{#1\hfill\break}
\newcommand{\CSLLeftMargin}[1]{\parbox[t]{\csllabelwidth}{#1}}
\newcommand{\CSLRightInline}[1]{\parbox[t]{\linewidth - \csllabelwidth}{#1}\break}
\newcommand{\CSLIndent}[1]{\hspace{\cslhangindent}#1}

\KOMAoption{captions}{tableheading}
\makeatletter
\makeatother
\makeatletter
\@ifpackageloaded{bookmark}{}{\usepackage{bookmark}}
\makeatother
\makeatletter
\@ifpackageloaded{caption}{}{\usepackage{caption}}
\AtBeginDocument{%
\ifdefined\contentsname
  \renewcommand*\contentsname{Table of contents}
\else
  \newcommand\contentsname{Table of contents}
\fi
\ifdefined\listfigurename
  \renewcommand*\listfigurename{List of Figures}
\else
  \newcommand\listfigurename{List of Figures}
\fi
\ifdefined\listtablename
  \renewcommand*\listtablename{List of Tables}
\else
  \newcommand\listtablename{List of Tables}
\fi
\ifdefined\figurename
  \renewcommand*\figurename{Fig}
\else
  \newcommand\figurename{Fig}
\fi
\ifdefined\tablename
  \renewcommand*\tablename{Tbl}
\else
  \newcommand\tablename{Tbl}
\fi
}
\@ifpackageloaded{float}{}{\usepackage{float}}
\floatstyle{ruled}
\@ifundefined{c@chapter}{\newfloat{codelisting}{h}{lop}}{\newfloat{codelisting}{h}{lop}[chapter]}
\floatname{codelisting}{Listing}
\newcommand*\listoflistings{\listof{codelisting}{List of Listings}}
\makeatother
\makeatletter
\@ifpackageloaded{caption}{}{\usepackage{caption}}
\@ifpackageloaded{subcaption}{}{\usepackage{subcaption}}
\makeatother
\makeatletter
\@ifpackageloaded{tcolorbox}{}{\usepackage[many]{tcolorbox}}
\makeatother
\makeatletter
\@ifundefined{shadecolor}{\definecolor{shadecolor}{rgb}{.97, .97, .97}}
\makeatother
\makeatletter
\@ifpackageloaded{sidenotes}{}{\usepackage{sidenotes}}
\@ifpackageloaded{marginnote}{}{\usepackage{marginnote}}
\makeatother
\makeatletter
\makeatother
\ifLuaTeX
  \usepackage{selnolig}  % disable illegal ligatures
\fi
\IfFileExists{bookmark.sty}{\usepackage{bookmark}}{\usepackage{hyperref}}
\IfFileExists{xurl.sty}{\usepackage{xurl}}{} % add URL line breaks if available
\urlstyle{same} % disable monospaced font for URLs
\hypersetup{
  pdftitle={Graviational Waves from Feynman Diagrams},
  pdfauthor={Lucien Huber},
  colorlinks=true,
  linkcolor={blue},
  filecolor={Maroon},
  citecolor={Blue},
  urlcolor={Blue},
  pdfcreator={LaTeX via pandoc}}

\title{Graviational Waves from Feynman Diagrams}
\author{Lucien Huber}
\date{9/13/2022}

\begin{document}
\maketitle

\newcommand{\mathtip}[2]{#1}
\newcommand{\texttip}[2]{#1}


\newcommand{\ind}[2]{^{#1}_{\phantom{#1}#2}} 	
\newcommand{\invind}[2]{_{#1}^{\phantom{#1}#2}}
\newcommand{\notipiunit}{{\mathring{\imath}}}
\newcommand{\iunit}{\mathtip{\notipiunit}{\text{Complex unit: } \notipiunit^2 = -1}}
\let\i\oldi
\newcommand{\i}{\iunit}

\let\exp\oldexp
\newcommand{\exp}[1]{\texttip{\mathrm{e}^{#1}}{exponential function}}
\newcommand{\Exp}[2][\paren]{\texttip{\exp#1{#2}}{exponential function}} 

\newcommand{\partialder}{\partial}
\newcommand{\differential}{\mathrm{d}}


\newcommand{\pdv}[3][]{\frac{\partialder^{#1} #2}{\partialder #3^{#1}}} 
\newcommand{\odv}[3][]{\frac{\differential^{#1} #2}{\differential #3^{#1}}}
\newcommand{\mdv}[3][]{\frac{\mathrm{D}^{#1} #2}{\mathrm{D} #3^{#1}}}
\newcommand{\fdv}[3][]{\frac{\delta^{#1} #2}{\delta #3^{#1}}}
\newcommand{\jdv}[3][]{\frac{\partialder ^{#1}(#2)}{\partialder (#3)^{#1}}}
\newcommand{\adv}[3][]{\frac{\Delta^{#1} #2}{\Delta #3^{#1}}}


\newcommand{\ipdv}[3][_]{#2\texttip{{}#1{,#3}}{Partial derivative}}

\newcommand{\covd}[3][_]{#2\texttip{{}#1{;#3}}{Covariant derivative}}

\newcommand{\half}[1][1]{\tfrac{#1}{2}}

\newcommand{\Half}[1][1]{\frac{#1}{2}}

\let\d\oldd
\newcommand{\d}[1]{\,\differential#1\,}
\newcommand{\dd}[2][]{\,\differential^{#1}#2\,}
\newcommand{\greens}[2][ret]{\texttip{\mathcal{G}_{\mathrm{#1}}\left(#2\right)}{#1 Green's function}}

\newcommand{\notipnormalized}[1]{\tilde{#1}}
\newcommand{\normalized}[2][]{\mathtip{\notipnormalized{#2}}{\text{Normalized } #1 =#2(2\pi)^{#1}}}


\newcommand{\dn}[2][]{\,\mathtip{\notipnormalized{\differential}^{#1}#2}{\notipnormalized{\differential}^{#1}#2=\frac{\dd[#1]{#2}}{(2\pi)^{#1}}}\,}

\newcommand{\thetafn}[2][]{\texttip{\Theta^{(#1)}\left(#2\right)}{Heaviside step function}}
\newcommand{\deltafn}[2][]{\texttip{\delta^{(#1)}\left(#2\right)}{Dirac delta function}}

\newcommand{\notipndeltafn}[2][]{\notipnormalized{\delta}^{(#1)}\left(#2\right)}
\newcommand{\ndeltafn}[2][]{\mathtip{\notipndeltafn[#1]{#2}}{\text 
{Normalized Dirac delta function: } \notipndeltafn[#1]{#2} = (2\pi)^{#1}\delta^{#1}\left(#2\right)}}

\newcommand{\notipposendelta}[2][-m^2]{\deltafn[+]{#2^2#1}}
\newcommand{\posendelta}[2][-m^2]{\mathtip{\notipposdelta[#1]{#2}}{\text{Positive energy delta function: }\notipposdelta[#1]{#2}}=\deltafn{#2^2 #1}\thetafn{#2^0}}

\newcommand{\notipnposendelta}[2][-m^2]{\notipndeltafn[+]{#2^2#1}}
\newcommand{\nposendelta}[2][-m^2]{\mathtip{\notipnposendelta[#1]{#2}}{\text{Normalized positive energy delta function: }\notipnposendelta[#1]{#2}=\notipposendelta[#1]{#2} 2\pi= 2 \pi \deltafn{#2^2 #1}\thetafn{#2^0}}}


\newcommand{\SRDA}{\mathtip{\Box_{SR}}{\invmink \partialder_\mu\partialder_\nu}}


\newcommand{\deltat}[1][\mu]{\texttip{\delta_{#1}}{delta function}}

\newcommand{\FT}[3][]{\mathtip{\mathcal{F}_{#1}\left[#2\right]}{\text{Fourier transform of $#2$ in variable $#1$: }{\mathcal{F}_{#1}\left[#2\right] = \int \dd{#1} #2(#1) \exp{\im#3 \cdot #1}   }}}

\newcommand{\invFT}[3][]{\mathtip{\mathcal{F}^{-1}_{#1}\left[#2\right]}{\text{inverse Fourier transform of $#2$ in variable $#1$: }{\mathcal{F}_{#1}\left[#2\right] = \frac{1}{2\pi}\int  \dd{#1}\exp{-\im#3 \cdot #1} #2(#1)  }}}

\newcommand{\ft}[3][x]{\mathtip{\hat{#2}}{\text{Fourier transform of $#2$ in variable $#1$: }{\hat{#2}\left(#3\right)
 = \int \dd{#1} #2(#1) \exp{\im#3 \cdot #1}}}}

\newcommand{\invft}[3][x]{\mathtip{\check{#2}}{\text{inverse Fourier transform of $#2$ in variable $#1$: }{\hat{#2}\left(#3\right) =\frac{1}{2\pi}\int  \dd{#1}\exp{-\im#3 \cdot #1} #2(#1)}}}
\newcommand{\ddP}[2][]{\,\mathtip{\differential\Phi_{#1}(#2)}{\text{On shell integration measure: } \differential\Phi(p_i)=\dn[4]{p_i}\nposendelta[-m_i^2]{p_i}}\,}
\newcommand{\AAngle}[1]{\Bigg\langle\hspace{-0.6em}\Bigg\langle #1\Bigg\rangle\hspace{-0.6em}\Bigg\rangle}


\newcommand{\F}{\mathbf{F}}
\newcommand{\x}{\mathbf{x}}
\newcommand{\y}{\mathbf{y}}
\newcommand{\z}{\mathbf{z}}
\let\k\oldk
\newcommand{\k}{\mathbf{k}}
\let\r\oldr
\newcommand{\r}{\mathbf{r}}
\let\p\oldp
\newcommand{\p}{\mathbf{p}}
\let\q\oldq
\let\v\oldv
\newcommand{\v}{\mathbf{v}}
\newcommand{\q}{\mathbf{q}}
\let\D\oldD
\newcommand{\D}{\mathrm{D}}
\newcommand{\im}{\iunit}

\let\L\oldL
\newcommand{\L}{\mathcal{L}}

\newcommand{\paren}[2][]{#1(#2 #1)}
\newcommand{\brk}[2][]{#1[#2 #1]}
\newcommand{\brc}[2][]{#1\{#2 #1\}}
\newcommand{\abs}[2][]{#1\vert#2 #1\vert}
\newcommand{\norm}[2][]{#1\Vert#2 #1\Vert}
\newcommand{\order}[2][]{\mathcal{O}\left(#2^{#1}\right)}

\let\sigma\sigmaold
\newcommand{\sigma}{{\mathtip{\href{index.html}{\sigmaold}}{\int\sin\d(x)}}}


\newcommand{\h}{\texttip{h}{planck's constant}}
\let\c\oldc
\newcommand{\c}{\texttip{c}{speed of light}}
\let\G\oldG
\newcommand{\G}{\texttip{G}{gravitational constant}}
\let\e\olde
\newcommand{\e}{\texttip{e}{electric charge}}
\let\m\oldm
\newcommand{\m}{\texttip{m}{mass}}

\newcommand{\metricTensor}{\texttip{g}{metric tensor}}
\newcommand{\minkoskiMetric}{\texttip{\eta}{Minkoski metric tensor +---}}
\newcommand{\mink}[1][{\mu\nu}]{\texttip{\minkoskiMetric_{#1}}{covariant Minkoski metric components +---}}
\newcommand{\invmink}[1][{\mu\nu}]{\texttip{\minkoskiMetric^{#1}}{contravariant Minkoski metric components +---}}



\newcommand{\met}[1][{\mu\nu}]{\texttip{\metricTensor_{#1}}{covariant metric tensor component}}
\newcommand{\invmet}[1][{\mu\nu}]{\texttip{\metricTensor^{#1}}{contravariant metric tensor component}}
\newcommand{\weakT}{\texttip{h}{weak metric tensor}}

\newcommand{\weakmet}[1][{\mu\nu}]{\texttip{\weakT_{#1}}{weak covariant metric tensor component}}
\newcommand{\weakinvmet}[1][{\mu\nu}]{\texttip{\weakT^{#1}}{weak contravariant metric tensor component}}
\newcommand{\trweakmet}[1][\alpha]{\texttip{\weakT\ind{#1}{#1}}{weak metric tensor trace}}
\newcommand{\weakmixmet}[1][\ind{\mu}{\nu}]{\texttip{\weakT#1}{weak mixed metric tensor component}} 

\newcommand{\weakTTR}{\texttip{\bar{h}}{trace reversed weak metric tensor}}
\newcommand{\weakTRmet}[1][\mu\nu]{\mathtip{\weakTTR_{#1}}{{\text{trace reversed weak covariant metric tensor component: } \weakTTR_{#1} = \weakmet-\frac{1}{2}\trweakmet\mink[#1]}}}
\newcommand{\weakTRinvmet}[1][{\mu\nu}]{\texttip{\weakTTR^{#1}}{{\text{trace reversed weak contravariant metric tensor component: } \weakTTR^{#1} = \weakmet-\frac{1}{2}\trweakmet\mink[#1]}}}
\newcommand{\trweakTRmet}[1][\alpha]{\texttip{\weakTTR\ind{#1}{#1}}{trace reversed weak metric tensor trace}}
\newcommand{\weakTRmixmet}[1][\ind{\mu}{\nu}]{\texttip{\weakTTR#1}{trace reversed weak mixed metric tensor component}}

\newcommand{\weakTTT}{\texttip{h^{TT}}{transverse traceless weak metric tensor}}
\newcommand{\weakTTmet}[1][ij]{\mathtip{\weakTTT_{#1}}{{\text{transverse traceless weak covariant metric tensor component: } \weakTTT_{0\mu}=0,\quad \weakTTT\ind{i}{i}=0, \quad \ipdv[^]{\weakTTT_{ij}}{j}=0.}}}
\newcommand{\weakTTinvmet}[1][{\mu\nu}]{\texttip{\weakTTT^{#1}}{{\text{transverse traceless weak contravariant metric tensor component: } \weakTTT_{0\mu}=0,\quad \weakTTT\ind{i}{i}=0, \quad \ipdv[^]{\weakTTT_{ij}}{j}=0.}}}
\newcommand{\trweakTTmet}[1][\alpha]{\texttip{\weakTTT\ind{#1}{#1}}{transverse traceless weak metric tensor trace}}
\newcommand{\weakTTmixmet}[1][\ind{\mu}{\nu}]{\texttip{\weakTTT#1}{transverse traceless weak mixed metric tensor component}}

\newcommand{\RiemannTensor}{\texttip{R}{Riemann tensor}}
\newcommand{\RiemannTensorExplicit}[4]{\ipdv{\LC{#1}{#2}{#3}}{#4} - \ipdv{\LC{#1}{#2}{#4}}{#3} - \LC{\alpha}{#2}{#4}\LC{#1}{\alpha}{#3}+\LC{\alpha}{#2}{#3}\LC{#1}{\alpha}{#4}}

\newcommand{\Riem}[1][\ind{\beta}{\mu\nu\rho}]{\texttip{\RiemannTensor#1}{Riemann Tensor tensor component}}
\newcommand{\RiemCt}[1][\beta\mu\nu\rho]{\texttip{\RiemannTensor_{#1}}{fully covariant Riemann tensor component}}

\newcommand{\RicciTensor}{\texttip{R}{Ricci tensor}}
\newcommand{\RicciTct}[1][{\mu\nu}]{\texttip{\RicciTensor^{#1}}{contravariant Ricci tensor component}}
\newcommand{\RicciTco}[1][{\mu\nu}]{\texttip{\RicciTensor_{#1}}{covariant Ricci tensor component}}

\newcommand{\linRicciTensor}{\texttip{R^{(1)}}{linearized Ricci tensor}}
\newcommand{\linRicciTct}[1][{\mu\nu}]{\texttip{\linRicciTensor^{#1}}{contravariant linearized Ricci tensor component}}
\newcommand{\linRicciTco}[1][{\mu\nu}]{\texttip{\linRicciTensor_{#1}}{covariant linearized Ricci tensor component}}

\newcommand{\RicciS}{\texttip{R}{Ricci scalar}}
\newcommand{\linRicciS}{\texttip{R^{(1)}}{linearized Ricci scalar}}

\newcommand{\SETensor}{\texttip{T}{Stress-Energy tensor, or energy-momentum tensor}}
\newcommand{\SEct}[1][{\mu\nu}]{\texttip{\SETensor^{#1}}{contravariant Stress-Energy tensor component}}
\newcommand{\SEco}[1][{\mu\nu}]{\texttip{\SETensor_{#1}}{covariant Stress-Energy tensor component}}
\newcommand{\trSE}[1][\alpha]{\texttip{\SETensor^{#1}{}_{#1}}{Stress-Energy tensor trace}}
\newcommand{\SEmix}[1][\ind{\mu}{\nu}]{\texttip{\SETensor#1}{mixed Stress-Energy tensor component}} 




\newcommand{\polTensor}{\texttip{\varepsilon}{Polarization tensor}}
\newcommand{\polct}[1][{\mu\nu}]{\texttip{\polTensor^{#1}}{contravariant Polarization tensor component}}
\newcommand{\polco}[1][{\mu\nu}]{\texttip{\polTensor_{#1}}{covariant Polarization tensor component}}
\newcommand{\trpol}[1][\alpha]{\texttip{\polTensor^{#1}{}_{#1}}{Polarization tensor trace}}
\newcommand{\polmix}[1][\ind{\mu}{\nu}]{\texttip{\polTensor#1}{mixed Polarization tensor component}}

\newcommand{\conj}[1]{\texttip{#1^\ast}{conjugate of #1}}

\newcommand{\conjpolTensor}{\texttip{\varepsilon^\ast}{Conjugate Polarization tensor}}
\newcommand{\conjpolct}[1][{\mu\nu}]{\texttip{\conjpolTensor^{#1}}{contravariant conjugate Polarization tensor component}}
\newcommand{\conjpolco}[1][{\mu\nu}]{\texttip{\conjpolTensor_{#1}}{covariant conjugate Polarization tensor component}}
\newcommand{\trconjpol}[1][\alpha]{\texttip{\conjpolTensor^{#1}{}_{#1}}{conjugate Polarization tensor trace}}
\newcommand{\conjpolmix}[1][\ind{\mu}{\nu}]{\texttip{\conjpolTensor#1}{mixed conjugate Polarization tensor component}}

\newcommand{\co}[2][\mu]{\texttip{#2_{#1}}{covariant mathbftor component}}
\newcommand{\ct}[2][\mu]{\texttip{#2^{#1}}{contravariant mathbftor component}}


\newcommand{\srcT}{\texttip{S}{source tensor}}
\newcommand{\srcco}[1][\mu\nu]{\texttip{\srcT_{#1}}{covariant source tensor component}}
\newcommand{\srcct}[1][\mu\nu]{\texttip{\srcT^{#1}}{contravariant source tensor component}}
\newcommand{\trsrc}[1][\alpha]{\texttip{\srcT^{#1}{}_{#1}}{source tensor trace}}
\newcommand{\srcmix}[1][\ind{\mu}{\nu}]{\texttip{\srcT#1}{mixed source tensor component}}

\newcommand{\kron}[1][_{\nu}^{\mu}]{\texttip{\delta#1}{Kronecker delta}}
\newcommand{\cosmo}{\texttip{\Lambda}{cosmological constant}}





\newcommand{\LC}[3]{\mathtip{\Gamma\ind{#1}{#2#3}}{\text{Levi-Civita Connection: }\Gamma\ind{#1}{#2#3}=
\frac{1}{2} \invmet[#1 \rho]
\brk[\Big]{
    \ipdv{\met[\rho #1]}{#2}
    +\ipdv{\met[\rho #2]}{#1}
    -\ipdv{\met[#2 #1]}{\rho}}
}} 

\let\bra\oldbra
\newcommand{\bra}[2][]{\mathtip{\sideset{_{\text{#1}}}{}\langle #2 \vert}{\text{Quantum state bra, dual to ket, a linear form on the Hilbert space: }#2: \mathcal{H} \rightarrow \mathbb{C}}}

\let\ket\oldket
\newcommand{\ket}[2][]{\mathtip{\sideset{}{_{\text{#1}}}{\vert #2 \rangle}}{\text{Quantum state ket, a vector of the Hilbert space: }#2\in\mathcal{H}}}


\newcommand{\notipobs}[1]{\mathcal{#1}}
\newcommand{\obs}[1]{\mathtip{\notipobs{#1}}{\text{Observable: }\notipobs{#1}}}

\newcommand{\notipop}[1]{\mathbb{#1}}
\newcommand{\op}[1]{\mathtip{\notipop{#1}}{\text{Operator: }\notipop{#1}}}

\newcommand{\notipev}[2]{\langle#2 \vert  #1 \vert #2 \rangle}
\newcommand{\ev}[2]{\mathtip{\notipev{#1}{#2}}{\text{Expectation value: }\notipev{#1}{#2}}}

\newcommand{\notipip}[2]{\langle #1 \vert #2 \rangle}
\newcommand{\ip}[2]{\mathtip{\notipip{#1}{#2}}{\text{Inner product: }\notipip{#1}{#2}}}

\newcommand{\notipcom}[2]{[#1,#2]}
\newcommand{\com}[2]{\mathtip{\notipcom{#1}{#2}}{\text{Commutator: }\notipcom{#1}{#2}}}

\newcommand{\notipmel}[3]{\langle #1 \vert #2 \vert #3 \rangle}
\newcommand{\mel}[3]{\mathtip{\notipmel{#1}{#2}{#3}}{\text{Matrix element: }\notipmel{#1}{#2}{#3}}}

\let\S\oldS
\newcommand{\S}{\texttip{S}{S-matrix}}
\let\T\oldT
\newcommand{\T}{\texttip{T}{Transfer-matrix}}


\newcommand{\notipvInt}[1][\obs{O}]{\mathcal{I}_{\text{v}}(#1)} 
\newcommand{\vInt}[1][\obs{O}]{\mathtip{\notipvInt[#1]}{\text{Virtual integrand: }\notipvInt[#1]}}

\newcommand{\notiprInt}[1][\obs{O}]{\mathcal{I}_{\text{r}}(#1)}
\newcommand{\rInt}[1][\obs{O}]{\mathtip{\notiprInt[#1]}{\text{Real integrand: }\notiprInt[#1]}}

\newcommand{\notipobschange}[3][\obs{O}]{\underset{#2-#3}{\Delta} \!#1\,}

\newcommand{\obschange}[3][\obs{O}]{\mathtip{\notipobschange[#1]{#2}{#3}}{\text{Observable change: }\notipobschange[#1]{#2}{#3}}}

\newcommand{\notipamp}[2]{\mathcal{A}(#1\to#2)}
\newcommand{\amp}[2]{\mathtip{\notipamp{#1}{#2}}{\text{Amplitude: }\notipamp{#1}{#2}}}

\ifdefined\Shaded\renewenvironment{Shaded}{\begin{tcolorbox}[borderline west={3pt}{0pt}{shadecolor}, breakable, interior hidden, boxrule=0pt, frame hidden, enhanced, sharp corners]}{\end{tcolorbox}}\fi

\renewcommand*\contentsname{Table of contents}
{
\hypersetup{linkcolor=}
\setcounter{tocdepth}{2}
\tableofcontents
}
\bookmarksetup{startatroot}

\hypertarget{preface}{%
\chapter*{Preface}\label{preface}}
\addcontentsline{toc}{chapter}{Preface}

This is a Quarto book. I am going to try and hyperlink it as much as
possible

To learn more about Quarto books visit
\url{https://quarto.org/docs/books}.

\bookmarksetup{startatroot}

\hypertarget{introduction}{%
\chapter{Introduction}\label{introduction}}

\begin{center}\rule{0.5\linewidth}{0.5pt}\end{center}

\begin{center}\rule{0.5\linewidth}{0.5pt}\end{center}

The detection of gravitational waves by the
Laser Interferometer Gravitational-Wave Observatory () and Virgo
collaborations in 2016
(2016)\marginpar{\begin{footnotesize}\leavevmode\vadjust pre{\protect\hypertarget{ref-LIGOScientific:2016aoc}{}}%
Abbott, B. P., R. Abbott, T. D. Abbott, et al. 2016. {``Observation of
{Gravitational Waves} from a {Binary Black Hole Merger}.''}
\emph{Physical Review Letters} 116 (6): 061102.
\url{https://doi.org/10.1103/PhysRevLett.116.061102}.\vspace{2mm}\par\end{footnotesize}}
has sparked a new era of gravitational wave astronomy. The first
detections were of Binary Black Hole () mergers. More recently,
Binary Neutron Star () mergers
(2017)\marginpar{\begin{footnotesize}\leavevmode\vadjust pre{\protect\hypertarget{ref-LIGOScientific:2017vwq}{}}%
Abbott, B. P., Rich Abbott, Thomas D. Abbott, et al. 2017.
{``{GW170817}: {Observation} of {Gravitational Waves} from a {Binary
Neutron Star Inspiral}.''} \emph{Physical Review Letters} 119 (16):
161101. \url{https://doi.org/10.1103/PhysRevLett.119.161101}.\vspace{2mm}\par\end{footnotesize}}
as well as Neutron Star ()-Black Hole () mergers have been detected
(2021)\marginpar{\begin{footnotesize}\leavevmode\vadjust pre{\protect\hypertarget{ref-LIGOScientific:2021qlt}{}}%
Abbott, R., T. D. Abbott, S. Abraham, F. Acernese, K. Ackley, A. Adams,
C. Adams, et al. 2021. {``Observation of {Gravitational Waves} from {Two
Neutron Star}\textendash{{Black Hole Coalescences}}.''} \emph{The
Astrophysical Journal Letters} 915 (1): L5.
\url{https://doi.org/10.3847/2041-8213/ac082e}.\vspace{2mm}\par\end{footnotesize}}.
Future detectors will further increase sensitivity and will be able to
detect a wide range of astrophysical sources. Studying these
gravitational waves signals gives us a very powerful new window into the
universe. It allows us to study the properties of neutron stars and
black holes, and the physics of compact object mergers, but also gives
us a powerful testing apparatus for general relativity.

To detect these faint signals {LIGO} and Virgo, have been made to be
extraordinarily precise instruments. Thus, they demand correspondingly
precise theoretical predictions and models. This is not just for
comparison's sake, but for detection as well. These faint signals are
often buried in the noise of the detector. To counteract this
experimental physicists make use of a matched filtering approach, where
they try to match the signal to a template. The template is a model of
the signal ideally provided by theoretical physicists based on physical
theories. The more precise the template, the higher the signal-to-noise
ratio, the more probable and precise the detection.

In this thesis we will explore the theoretical landscape surrounding the
generation of these templates. We will focus on a nascent subfield where
tools originally used for Quantum Field Theory () and particle physics
are being applied to the study of gravitational waves. Specifically we
are interested in the diagrammatic objects that arise when framing the
two body problem similarly to particle collisions. We will explain where
these tools shine in the broader context of waveform approaches such as
Effective One-Body (), Nonrelativistic General Relativity (),
Post-Minkowski () and Post-Newtonian () approximations. We will also
discuss the challenges that lie ahead in the field.

\bookmarksetup{startatroot}

\hypertarget{gravitational-wave-generation}{%
\chapter{Gravitational Wave
Generation}\label{gravitational-wave-generation}}

First let us look at where and how gravitional waves are generated. We
will look at how General Relativity () predicts that graviational waves
exist, and what conditions have to be met for them to become observable.
We will also look at how we can detect them, and what we can learn from
them.

\hypertarget{gravitational-waves}{%
\section{Gravitational Waves}\label{gravitational-waves}}

The most complete theory of gravity we have right now is that of {GR},
due to Einstein. It formulates spacetime as a Riemannian manifold with
curvature, senstive to mass. Objects then move around in that deformed
spacetime encoded in the metric
\(\texttip{\texttip{g}{metric tensor}_{{\mu\nu}}}{covariant metric tensor component}\).
Objects interact gravitationally when the spacetime they move around in
is affected by the curvature caused by other objects. The equation that
governs this interaction between mass and curvature is Einstein's field
equations,

\begin{equation}\protect\hypertarget{eq-EFE}{}{
\boxed{\texttip{\texttip{R}{Ricci tensor}_{{\mu\nu}}}{covariant Ricci tensor component}-\frac{\texttip{\texttip{g}{metric tensor}_{{\mu\nu}}}{covariant metric tensor component}}{2} \texttip{R}{Ricci scalar}+\texttip{\Lambda}{cosmological constant}\texttip{\texttip{g}{metric tensor}_{{\mu\nu}}}{covariant metric tensor component}=-8 \pi \oldG\texttip{\texttip{T}{Stress-Energy tensor, or energy-momentum tensor}_{{\mu\nu}}}{covariant Stress-Energy tensor component}} .
}\label{eq-EFE}\end{equation}

The Left-Hand Side () of this equation, contains several objects that
are all only really dependent on the metric
\(\texttip{\texttip{g}{metric tensor}_{{\mu\nu}}}{covariant metric tensor component}\).
\(\texttip{\texttip{R}{Ricci tensor}_{{\mu\nu}}}{covariant Ricci tensor component}\)
is the Ricci tensor\sidenote{\footnotesize The Ricci Tensor is given by: \[
  \texttip{\texttip{R}{Ricci tensor}_{{\mu\nu}}}{covariant Ricci tensor component}\coloneq \texttip{\texttip{R}{Riemann tensor}^{\alpha}_{\phantom{\alpha}\mu\alpha\nu}}{Riemann Tensor tensor component}=\texttip{\texttip{g}{metric tensor}^{\alpha\beta}}{contravariant metric tensor component}\texttip{\texttip{R}{Riemann tensor}_{\alpha\mu\beta\nu}}{fully covariant Riemann tensor component}
  \]}, a contraction of the Riemann curvature tensor
\(\texttip{\texttip{R}{Riemann tensor}^{\beta}_{\phantom{\beta}\mu\nu\rho}}{Riemann Tensor tensor component}\).
This tensor\sidenote{\footnotesize The Riemann Tensor is given by: \[
  \texttip{\texttip{R}{Riemann tensor}^{\beta}_{\phantom{\beta}\mu\nu\rho}}{Riemann Tensor tensor component}=\mathtip{\Gamma^{\beta}_{\phantom{\beta}\mu\nu}}{\text{Levi-Civita Connection: }\Gamma^{\beta}_{\phantom{\beta}\mu\nu}=
  \frac{1}{2} \texttip{\texttip{g}{metric tensor}^{\beta \rho}}{contravariant metric tensor component}
  \Big[
      \texttip{\texttip{g}{metric tensor}_{\rho \beta}}{covariant metric tensor component}\texttip{{}_{,\mu}}{Partial derivative}
      +\texttip{\texttip{g}{metric tensor}_{\rho \mu}}{covariant metric tensor component}\texttip{{}_{,\beta}}{Partial derivative}
      -\texttip{\texttip{g}{metric tensor}_{\mu \beta}}{covariant metric tensor component}\texttip{{}_{,\rho}}{Partial derivative} \Big]
  }\texttip{{}_{,\rho}}{Partial derivative} - \mathtip{\Gamma^{\beta}_{\phantom{\beta}\mu\rho}}{\text{Levi-Civita Connection: }\Gamma^{\beta}_{\phantom{\beta}\mu\rho}=
  \frac{1}{2} \texttip{\texttip{g}{metric tensor}^{\beta \rho}}{contravariant metric tensor component}
  \Big[
      \texttip{\texttip{g}{metric tensor}_{\rho \beta}}{covariant metric tensor component}\texttip{{}_{,\mu}}{Partial derivative}
      +\texttip{\texttip{g}{metric tensor}_{\rho \mu}}{covariant metric tensor component}\texttip{{}_{,\beta}}{Partial derivative}
      -\texttip{\texttip{g}{metric tensor}_{\mu \beta}}{covariant metric tensor component}\texttip{{}_{,\rho}}{Partial derivative} \Big]
  }\texttip{{}_{,\nu}}{Partial derivative} - \mathtip{\Gamma^{\alpha}_{\phantom{\alpha}\mu\rho}}{\text{Levi-Civita Connection: }\Gamma^{\alpha}_{\phantom{\alpha}\mu\rho}=
  \frac{1}{2} \texttip{\texttip{g}{metric tensor}^{\alpha \rho}}{contravariant metric tensor component}
  \Big[
      \texttip{\texttip{g}{metric tensor}_{\rho \alpha}}{covariant metric tensor component}\texttip{{}_{,\mu}}{Partial derivative}
      +\texttip{\texttip{g}{metric tensor}_{\rho \mu}}{covariant metric tensor component}\texttip{{}_{,\alpha}}{Partial derivative}
      -\texttip{\texttip{g}{metric tensor}_{\mu \alpha}}{covariant metric tensor component}\texttip{{}_{,\rho}}{Partial derivative} \Big]
  }\mathtip{\Gamma^{\beta}_{\phantom{\beta}\alpha\nu}}{\text{Levi-Civita Connection: }\Gamma^{\beta}_{\phantom{\beta}\alpha\nu}=
  \frac{1}{2} \texttip{\texttip{g}{metric tensor}^{\beta \rho}}{contravariant metric tensor component}
  \Big[
      \texttip{\texttip{g}{metric tensor}_{\rho \beta}}{covariant metric tensor component}\texttip{{}_{,\alpha}}{Partial derivative}
      +\texttip{\texttip{g}{metric tensor}_{\rho \alpha}}{covariant metric tensor component}\texttip{{}_{,\beta}}{Partial derivative}
      -\texttip{\texttip{g}{metric tensor}_{\alpha \beta}}{covariant metric tensor component}\texttip{{}_{,\rho}}{Partial derivative} \Big]
  }+\mathtip{\Gamma^{\alpha}_{\phantom{\alpha}\mu\nu}}{\text{Levi-Civita Connection: }\Gamma^{\alpha}_{\phantom{\alpha}\mu\nu}=
  \frac{1}{2} \texttip{\texttip{g}{metric tensor}^{\alpha \rho}}{contravariant metric tensor component}
  \Big[
      \texttip{\texttip{g}{metric tensor}_{\rho \alpha}}{covariant metric tensor component}\texttip{{}_{,\mu}}{Partial derivative}
      +\texttip{\texttip{g}{metric tensor}_{\rho \mu}}{covariant metric tensor component}\texttip{{}_{,\alpha}}{Partial derivative}
      -\texttip{\texttip{g}{metric tensor}_{\mu \alpha}}{covariant metric tensor component}\texttip{{}_{,\rho}}{Partial derivative} \Big]
  }\mathtip{\Gamma^{\beta}_{\phantom{\beta}\alpha\rho}}{\text{Levi-Civita Connection: }\Gamma^{\beta}_{\phantom{\beta}\alpha\rho}=
  \frac{1}{2} \texttip{\texttip{g}{metric tensor}^{\beta \rho}}{contravariant metric tensor component}
  \Big[
      \texttip{\texttip{g}{metric tensor}_{\rho \beta}}{covariant metric tensor component}\texttip{{}_{,\alpha}}{Partial derivative}
      +\texttip{\texttip{g}{metric tensor}_{\rho \alpha}}{covariant metric tensor component}\texttip{{}_{,\beta}}{Partial derivative}
      -\texttip{\texttip{g}{metric tensor}_{\alpha \beta}}{covariant metric tensor component}\texttip{{}_{,\rho}}{Partial derivative} \Big]
  }.
  \]} encodes the curvature of spacetime, as it is essentially a
measurement of the amount with which covariant derivatives don't
commute. When they do, it means that they have collapsed to regular
derivatives, and thus the Levi-civita connection\sidenote{\footnotesize The
  Levi-Civita Connection is given by:
  \[\mathtip{\Gamma^{\alpha}_{\phantom{\alpha}\mu\nu}}{\text{Levi-Civita Connection: }\Gamma^{\alpha}_{\phantom{\alpha}\mu\nu}=
  \frac{1}{2} \texttip{\texttip{g}{metric tensor}^{\alpha \rho}}{contravariant metric tensor component}
  \Big[
      \texttip{\texttip{g}{metric tensor}_{\rho \alpha}}{covariant metric tensor component}\texttip{{}_{,\mu}}{Partial derivative}
      +\texttip{\texttip{g}{metric tensor}_{\rho \mu}}{covariant metric tensor component}\texttip{{}_{,\alpha}}{Partial derivative}
      -\texttip{\texttip{g}{metric tensor}_{\mu \alpha}}{covariant metric tensor component}\texttip{{}_{,\rho}}{Partial derivative} \Big]
  }=
  \frac{1}{2} \texttip{\texttip{g}{metric tensor}^{\alpha \rho}}{contravariant metric tensor component}
  \Big[
   \texttip{\texttip{g}{metric tensor}_{\rho \nu}}{covariant metric tensor component}\texttip{{}_{,\mu}}{Partial derivative}
  + \texttip{\texttip{g}{metric tensor}_{\rho \mu}}{covariant metric tensor component}\texttip{{}_{,\nu}}{Partial derivative}
  - \texttip{\texttip{g}{metric tensor}_{{\mu\nu}}}{covariant metric tensor component}\texttip{{}_{,\rho}}{Partial derivative} \Big]
  \]} must have vanished. This is only possible if the metric is flat
\(\texttip{\texttip{\eta}{Minkoski metric tensor +---}_{{\mu\nu}}}{covariant Minkoski metric components +---}\).
The Riemann curvature tensor is exclusively made up of the metric, its
first, second derivatives and it is linear in the second derivative of
the metric. In fact it is the only possible tensor of that sort. The
second term on the {LHS} is the Ricci scalar, a further contraction of
the Ricci Tensor:
\(\texttip{R}{Ricci scalar}= \texttip{\texttip{g}{metric tensor}^{{\mu\nu}}}{contravariant metric tensor component}\texttip{\texttip{R}{Ricci tensor}_{{\mu\nu}}}{covariant Ricci tensor component}\).
The final term on the {LHS} is just a proportional constant factor of
the metric, where \(\texttip{\Lambda}{cosmological constant}\) is called
the cosmological constant. It can be measured and has a low known upper
bound. It is in part responsible for the expansion of the universe.

The Right-Hand Side () of \ref{eq-EFE} encodes the effect of mass and
energy on the metric.
\(\texttip{\texttip{T}{Stress-Energy tensor, or energy-momentum tensor}_{{\mu\nu}}}{covariant Stress-Energy tensor component}\)
is the stress-energy tensor, dependent on the dynamics of the system. In
empty space this term is zero. It is further multiplied by the
Gravitational constant \(G\).

Equation \ref{eq-EFE} can be recast in a form where the Ricci scalar has
been eliminated,

\begin{equation}\protect\hypertarget{eq-EFEwoR}{}{
\boxed{\texttip{\texttip{R}{Ricci tensor}_{{\mu\nu}}}{covariant Ricci tensor component}=-8 \pi G\left(\texttip{\texttip{T}{Stress-Energy tensor, or energy-momentum tensor}_{{\mu\nu}}}{covariant Stress-Energy tensor component}-\texttip{\texttip{T}{Stress-Energy tensor, or energy-momentum tensor}^{\alpha}{}_{\alpha}}{Stress-Energy tensor trace}\frac{\texttip{\texttip{g}{metric tensor}_{{\mu\nu}}}{covariant metric tensor component}}{2}\right)+\texttip{\Lambda}{cosmological constant}\texttip{\texttip{g}{metric tensor}_{{\mu\nu}}}{covariant metric tensor component}} 
}\label{eq-EFEwoR}\end{equation}

Either equation \ref{eq-EFE}, \ref{eq-EFEwoR} with
\(\texttip{\Lambda}{cosmological constant}=0\) admit wave solutions. We
can see this by looking at these equations in the weak field
approximation. Namely, we take the metric to be a Minkowskian background
and a small perturbation

\begin{equation}\protect\hypertarget{eq-WeakFA}{}{
    \texttip{\texttip{g}{metric tensor}_{{\mu\nu}}}{covariant metric tensor component}= \texttip{\texttip{\eta}{Minkoski metric tensor +---}_{{\mu\nu}}}{covariant Minkoski metric components +---}+ \texttip{\texttip{h}{weak metric tensor}_{{\mu\nu}}}{weak covariant metric tensor component},
}\label{eq-WeakFA}\end{equation}

with
\(\texttip{\texttip{h}{weak metric tensor}_{{\mu\nu}}}{weak covariant metric tensor component}\)
small. Note that if we restrict ourselves to first order in
\(\texttip{\texttip{h}{weak metric tensor}_{{\mu\nu}}}{weak covariant metric tensor component}\)
then all raising and lowering of indices has to be done with
\(\texttip{\texttip{\eta}{Minkoski metric tensor +---}_{{\mu\nu}}}{covariant Minkoski metric components +---}\),
or else we increase the order of the term by 1. Now since the {LHS} of
\ref{eq-EFE} is made up of the Ricci tensor and scalar, let us see how
these behave in the weak approximation. Both are in fact made up of
Levi-Civita connections, which to first order in
\(\texttip{\texttip{h}{weak metric tensor}_{{\mu\nu}}}{weak covariant metric tensor component}\)
is given by:
\begin{equation}\protect\hypertarget{eq-weakaffine}{}{\mathtip{\Gamma^{\alpha}_{\phantom{\alpha}\mu\nu}}{\text{Levi-Civita Connection: }\Gamma^{\alpha}_{\phantom{\alpha}\mu\nu}=
\frac{1}{2} \texttip{\texttip{g}{metric tensor}^{\alpha \rho}}{contravariant metric tensor component}
\Big[
    \texttip{\texttip{g}{metric tensor}_{\rho \alpha}}{covariant metric tensor component}\texttip{{}_{,\mu}}{Partial derivative}
    +\texttip{\texttip{g}{metric tensor}_{\rho \mu}}{covariant metric tensor component}\texttip{{}_{,\alpha}}{Partial derivative}
    -\texttip{\texttip{g}{metric tensor}_{\mu \alpha}}{covariant metric tensor component}\texttip{{}_{,\rho}}{Partial derivative} \Big]
} = \frac{1}{2} \texttip{\texttip{\eta}{Minkoski metric tensor +---}^{\alpha \rho}}{contravariant Minkoski metric components +---}
\Big[
     \texttip{\texttip{h}{weak metric tensor}_{\rho \nu}}{weak covariant metric tensor component}\texttip{{}_{,\mu}}{Partial derivative}
    + \texttip{\texttip{h}{weak metric tensor}_{\rho \mu}}{weak covariant metric tensor component}\texttip{{}_{,\nu}}{Partial derivative}
    - \texttip{\texttip{h}{weak metric tensor}_{{\mu\nu}}}{weak covariant metric tensor component}\texttip{{}_{,\rho}}{Partial derivative} \Big] + \mathcal{O}\left(h^{2}\right)\quad .}\label{eq-weakaffine}\end{equation}

Plugging in to the definition of the Ricci tensor we see that the terms
with products of the connection vanish, and we are left with the
derivative terms:

\begin{equation}\protect\hypertarget{eq-Ricciweak}{}{ 
\texttip{\texttip{R}{Ricci tensor}_{{\mu\nu}}}{covariant Ricci tensor component}=
\toggle{ \frac{\texttip{\texttip{\eta}{Minkoski metric tensor +---}^{\alpha\rho}}{contravariant Minkoski metric components +---}}{2}\left[ \texttip{\texttip{h}{weak metric tensor}_{\rho\alpha}}{weak covariant metric tensor component}\texttip{{}_{,\mu\nu}}{Partial derivative} + \texttip{\texttip{h}{weak metric tensor}_{{\mu\nu}}}{weak covariant metric tensor component}\texttip{{}_{,\rho\alpha}}{Partial derivative} - \texttip{\texttip{h}{weak metric tensor}_{\mu\alpha}}{weak covariant metric tensor component}\texttip{{}_{,\rho\nu}}{Partial derivative} - \texttip{\texttip{h}{weak metric tensor}_{\nu\rho}}{weak covariant metric tensor component}\texttip{{}_{,\mu\alpha}}{Partial derivative} \right]}{\texttip{\texttip{R^{(1)}}{linearized Ricci tensor}_{{\mu\nu}}}{covariant linearized Ricci tensor component}}\endtoggle + \mathcal{O}(h^2)=\texttip{\texttip{R^{(1)}}{linearized Ricci tensor}_{{\mu\nu}}}{covariant linearized Ricci tensor component}+ \mathcal{O}(h^2)
}\label{eq-Ricciweak}\end{equation}

The linearized Ricci scalar is then just
\(\texttip{R^{(1)}}{linearized Ricci scalar}=\texttip{\texttip{\eta}{Minkoski metric tensor +---}^{{\mu\nu}}}{contravariant Minkoski metric components +---}\texttip{\texttip{R^{(1)}}{linearized Ricci tensor}_{{\mu\nu}}}{covariant linearized Ricci tensor component}\).
Regardless of the weak approximation, \ref{eq-EFE} has some gauge
freedom. This means that if we solve eq \ref{eq-EFE}, it won't be the
only possible solution, and in fact we can generate the others by
changing coordinated in such a way that the equation isn't effected. To
fix this ambiguity we choose a coordinate system (`gauge'), by imposing
the harmonic coordinate conditions:

\begin{equation}\protect\hypertarget{eq-hcc}{}{ 
   \texttip{\texttip{g}{metric tensor}^{\alpha\beta}}{contravariant metric tensor component}\mathtip{\Gamma^{\mu}_{\phantom{\mu}\alpha\beta}}{\text{Levi-Civita Connection: }\Gamma^{\mu}_{\phantom{\mu}\alpha\beta}=
\frac{1}{2} \texttip{\texttip{g}{metric tensor}^{\mu \rho}}{contravariant metric tensor component}
\Big[
    \texttip{\texttip{g}{metric tensor}_{\rho \mu}}{covariant metric tensor component}\texttip{{}_{,\alpha}}{Partial derivative}
    +\texttip{\texttip{g}{metric tensor}_{\rho \alpha}}{covariant metric tensor component}\texttip{{}_{,\mu}}{Partial derivative}
    -\texttip{\texttip{g}{metric tensor}_{\alpha \mu}}{covariant metric tensor component}\texttip{{}_{,\rho}}{Partial derivative} \Big]
} =0
}\label{eq-hcc}\end{equation}

The harmonic coordinate conditions demand the vanishing of the
Levi-Civita connection. They have a simplified form in the weak
approximation, which we can acces by plugging \ref{eq-weakaffine} into
\ref{eq-hcc}, giving: \[
\begin{split}
    (\texttip{\texttip{\eta}{Minkoski metric tensor +---}^{\alpha\beta}}{contravariant Minkoski metric components +---}+\texttip{\texttip{h}{weak metric tensor}^{\alpha\beta}}{weak contravariant metric tensor component})\frac{1}{2}(\texttip{\texttip{\eta}{Minkoski metric tensor +---}^{\mu\rho}}{contravariant Minkoski metric components +---}+\texttip{\texttip{h}{weak metric tensor}^{\mu\rho}}{weak contravariant metric tensor component})\left[\texttip{\texttip{h}{weak metric tensor}_{\alpha\rho}}{weak covariant metric tensor component}\texttip{{}_{,\beta}}{Partial derivative}+\texttip{\texttip{h}{weak metric tensor}_{\beta\rho}}{weak covariant metric tensor component}\texttip{{}_{,\alpha}}{Partial derivative}-\texttip{\texttip{h}{weak metric tensor}_{\alpha\beta}}{weak covariant metric tensor component}\texttip{{}_{,\rho}}{Partial derivative}\right] &= 0 \\
    \texttip{\texttip{\eta}{Minkoski metric tensor +---}^{\mu\rho}}{contravariant Minkoski metric components +---}\texttip{\texttip{\eta}{Minkoski metric tensor +---}^{\alpha\beta}}{contravariant Minkoski metric components +---}\left[2\texttip{\texttip{h}{weak metric tensor}_{\alpha\rho}}{weak covariant metric tensor component}\texttip{{}_{,\beta}}{Partial derivative}-\texttip{\texttip{h}{weak metric tensor}_{\alpha\beta}}{weak covariant metric tensor component}\texttip{{}_{,\rho}}{Partial derivative}\right]+ \mathcal{O}(h^2)&= 0 \\
    \texttip{\texttip{\eta}{Minkoski metric tensor +---}^{\alpha\beta}}{contravariant Minkoski metric components +---}\texttip{\texttip{h}{weak metric tensor}_{\alpha\rho}}{weak covariant metric tensor component}\texttip{{}_{,\beta}}{Partial derivative} &= \frac{1}{2}\texttip{\texttip{h}{weak metric tensor}_{\alpha\beta}}{weak covariant metric tensor component}\texttip{{}_{,\rho}}{Partial derivative}\texttip{\texttip{\eta}{Minkoski metric tensor +---}^{\alpha\beta}}{contravariant Minkoski metric components +---} + \mathcal{O}(h^2).
\end{split}
\]

The last equation to first order is called the de Donder gauge and is
often written: \[
\texttip{\texttip{h}{weak metric tensor}_{{\mu\nu}}}{weak covariant metric tensor component}\texttip{{}^{,\mu}}{Partial derivative} = \frac{1}{2}\texttip{\texttip{h}{weak metric tensor}^{\alpha}_{\phantom{\alpha}\alpha}}{weak metric tensor trace}\texttip{{}_{,\nu}}{Partial derivative}
\] In de Donder gauge we can simplify the terms in \ref{eq-Ricciweak}
to:

\[
    \texttip{\texttip{\eta}{Minkoski metric tensor +---}^{\alpha\beta}}{contravariant Minkoski metric components +---}\texttip{\texttip{h}{weak metric tensor}_{\alpha\mu}}{weak covariant metric tensor component}\texttip{{}_{,\nu\beta}}{Partial derivative} \approx \frac{1}{2}\texttip{\texttip{\eta}{Minkoski metric tensor +---}^{\alpha\beta}}{contravariant Minkoski metric components +---}\texttip{\texttip{h}{weak metric tensor}_{\alpha\beta}}{weak covariant metric tensor component}\texttip{{}_{,\mu\nu}}{Partial derivative},
\] and \[
    \texttip{\texttip{\eta}{Minkoski metric tensor +---}^{\alpha\beta}}{contravariant Minkoski metric components +---}\texttip{\texttip{h}{weak metric tensor}_{\beta\nu}}{weak covariant metric tensor component}\texttip{{}_{,\mu\alpha}}{Partial derivative} \approx \frac{1}{2}\texttip{\texttip{h}{weak metric tensor}_{\alpha\beta}}{weak covariant metric tensor component}\texttip{{}_{,\mu\nu}}{Partial derivative}.
\] With these two relations the expression for the linearized Ricci
tensor \ref{eq-Ricciweak} simplifies to
\begin{equation}\protect\hypertarget{eq-Ricciweak2}{}{
    \texttip{\texttip{R^{(1)}}{linearized Ricci tensor}_{{\mu\nu}}}{covariant linearized Ricci tensor component}= \frac{1}{2}\texttip{\texttip{\eta}{Minkoski metric tensor +---}^{\alpha\beta}}{contravariant Minkoski metric components +---}\texttip{\texttip{h}{weak metric tensor}_{{\mu\nu}}}{weak covariant metric tensor component}\texttip{{}_{,\alpha\beta}}{Partial derivative} = \frac{1}{2}\mathtip{\Box_{SR}}{\texttip{\texttip{\eta}{Minkoski metric tensor +---}^{{\mu\nu}}}{contravariant Minkoski metric components +---}\partial_\mu\partial_\nu}\texttip{\texttip{h}{weak metric tensor}_{{\mu\nu}}}{weak covariant metric tensor component}.
}\label{eq-Ricciweak2}\end{equation}\\
Where we have defined the Special Relativity () D'Alembertian as:
\(\mathtip{\Box_{SR}}{\texttip{\texttip{\eta}{Minkoski metric tensor +---}^{{\mu\nu}}}{contravariant Minkoski metric components +---}\partial_\mu\partial_\nu}=\texttip{\texttip{\eta}{Minkoski metric tensor +---}^{{\mu\nu}}}{contravariant Minkoski metric components +---}\partial_\mu\partial_\nu\)
We can plug this into \ref{eq-EFEwoR}, with
\(\texttip{\Lambda}{cosmological constant}=0\), and up to first order in
\(\texttip{\texttip{h}{weak metric tensor}_{{\mu\nu}}}{weak covariant metric tensor component}\)
we get the linearized Einstein field equations for a system of harmonic
coordinates:

\begin{equation}\protect\hypertarget{eq-linearized}{}{
\square_{SR} \texttip{\texttip{h}{weak metric tensor}_{{\mu\nu}}}{weak covariant metric tensor component}= -16\pi \oldG\overbracket{(\texttip{\texttip{T}{Stress-Energy tensor, or energy-momentum tensor}_{{\mu\nu}}}{covariant Stress-Energy tensor component}-\frac{\texttip{\texttip{\eta}{Minkoski metric tensor +---}_{{\mu\nu}}}{covariant Minkoski metric components +---}}{2}\texttip{\texttip{T}{Stress-Energy tensor, or energy-momentum tensor}^{\alpha}{}_{\alpha}}{Stress-Energy tensor trace})}^{\texttip{\texttip{S}{source tensor}_{\mu\nu}}{covariant source tensor component}} 
}\label{eq-linearized}\end{equation}
\begin{equation}\protect\hypertarget{eq-dedonder}{}{
\texttip{\texttip{h}{weak metric tensor}^{\alpha}_{\phantom{\alpha}\mu}}{weak mixed metric tensor component}\texttip{{}_{,\alpha}}{Partial derivative}=\frac{1}{2}\texttip{\texttip{h}{weak metric tensor}^{\alpha}_{\phantom{\alpha}\alpha}}{weak metric tensor trace}\texttip{{}_{,\mu}}{Partial derivative}. 
}\label{eq-dedonder}\end{equation} Where raising and lowering indices
has been done with the Minkowski metric. The tensor
\(\texttip{\texttip{S}{source tensor}_{\mu\nu}}{covariant source tensor component}\)
encodes the behavior of the source of gravitional waves. One could also
plug \ref{eq-Ricciweak2} into \ref{eq-EFE}, with
\(\texttip{\Lambda}{cosmological constant}=0\), and if we change to the
trace reversed perturbation:
\(\mathtip{\texttip{\bar{h}}{trace reversed weak metric tensor}_{\mu\nu}}{{\text{trace reversed weak covariant metric tensor component: } \texttip{\bar{h}}{trace reversed weak metric tensor}_{\mu\nu} = \texttip{\texttip{h}{weak metric tensor}_{{\mu\nu}}}{weak covariant metric tensor component}-\frac{1}{2}\texttip{\texttip{h}{weak metric tensor}^{\alpha}_{\phantom{\alpha}\alpha}}{weak metric tensor trace}\texttip{\texttip{\eta}{Minkoski metric tensor +---}_{\mu\nu}}{covariant Minkoski metric components +---}}}=\texttip{\texttip{h}{weak metric tensor}_{{\mu\nu}}}{weak covariant metric tensor component}- \frac{1}{2}\texttip{\texttip{h}{weak metric tensor}^{\alpha}_{\phantom{\alpha}\alpha}}{weak metric tensor trace}\texttip{\texttip{\eta}{Minkoski metric tensor +---}_{{\mu\nu}}}{covariant Minkoski metric components +---}\),
we get similar and simpler equations at the cost of a more complex
perturbation: \sidenote{\footnotesize Note that the inverse change of variables is
  just:
  \(\texttip{\texttip{h}{weak metric tensor}_{{\mu\nu}}}{weak covariant metric tensor component}=\mathtip{\texttip{\bar{h}}{trace reversed weak metric tensor}_{\mu\nu}}{{\text{trace reversed weak covariant metric tensor component: } \texttip{\bar{h}}{trace reversed weak metric tensor}_{\mu\nu} = \texttip{\texttip{h}{weak metric tensor}_{{\mu\nu}}}{weak covariant metric tensor component}-\frac{1}{2}\texttip{\texttip{h}{weak metric tensor}^{\alpha}_{\phantom{\alpha}\alpha}}{weak metric tensor trace}\texttip{\texttip{\eta}{Minkoski metric tensor +---}_{\mu\nu}}{covariant Minkoski metric components +---}}}-\frac{1}{2}\texttip{\texttip{\bar{h}}{trace reversed weak metric tensor}^{\alpha}_{\phantom{\alpha}\alpha}}{trace reversed weak metric tensor trace}\texttip{\texttip{\eta}{Minkoski metric tensor +---}_{{\mu\nu}}}{covariant Minkoski metric components +---}\).}
\sidenote{\footnotesize We eliminate the trace of the stress-energy tensor by using:
  \(\texttip{R}{Ricci scalar}= 8\pi\oldG\texttip{\texttip{T}{Stress-Energy tensor, or energy-momentum tensor}^{\alpha}{}_{\alpha}}{Stress-Energy tensor trace}\).
  We can write
  \(\texttip{R^{(1)}}{linearized Ricci scalar}=-\frac{1}{2}\mathtip{\Box_{SR}}{\texttip{\texttip{\eta}{Minkoski metric tensor +---}^{{\mu\nu}}}{contravariant Minkoski metric components +---}\partial_\mu\partial_\nu}\texttip{\texttip{\bar{h}}{trace reversed weak metric tensor}^{\alpha}_{\phantom{\alpha}\alpha}}{trace reversed weak metric tensor trace}\),
  in de Donder gauge, which is precisely the extra term dropping out of
  \(\mathtip{\Box_{SR}}{\texttip{\texttip{\eta}{Minkoski metric tensor +---}^{{\mu\nu}}}{contravariant Minkoski metric components +---}\partial_\mu\partial_\nu}\texttip{\texttip{h}{weak metric tensor}_{{\mu\nu}}}{weak covariant metric tensor component}\)
  when we express it in terms of
  \(\mathtip{\texttip{\bar{h}}{trace reversed weak metric tensor}_{\mu\nu}}{{\text{trace reversed weak covariant metric tensor component: } \texttip{\bar{h}}{trace reversed weak metric tensor}_{\mu\nu} = \texttip{\texttip{h}{weak metric tensor}_{{\mu\nu}}}{weak covariant metric tensor component}-\frac{1}{2}\texttip{\texttip{h}{weak metric tensor}^{\alpha}_{\phantom{\alpha}\alpha}}{weak metric tensor trace}\texttip{\texttip{\eta}{Minkoski metric tensor +---}_{\mu\nu}}{covariant Minkoski metric components +---}}}\)}

\begin{equation}\protect\hypertarget{eq-linearizedTR}{}{
\square_{SR} \mathtip{\texttip{\bar{h}}{trace reversed weak metric tensor}_{\mu\nu}}{{\text{trace reversed weak covariant metric tensor component: } \texttip{\bar{h}}{trace reversed weak metric tensor}_{\mu\nu} = \texttip{\texttip{h}{weak metric tensor}_{{\mu\nu}}}{weak covariant metric tensor component}-\frac{1}{2}\texttip{\texttip{h}{weak metric tensor}^{\alpha}_{\phantom{\alpha}\alpha}}{weak metric tensor trace}\texttip{\texttip{\eta}{Minkoski metric tensor +---}_{\mu\nu}}{covariant Minkoski metric components +---}}}= -16\pi \oldG\texttip{\texttip{T}{Stress-Energy tensor, or energy-momentum tensor}_{{\mu\nu}}}{covariant Stress-Energy tensor component}
}\label{eq-linearizedTR}\end{equation}
\begin{equation}\protect\hypertarget{eq-dedonderTR}{}{
    \mathtip{\texttip{\bar{h}}{trace reversed weak metric tensor}_{\mu\nu}}{{\text{trace reversed weak covariant metric tensor component: } \texttip{\bar{h}}{trace reversed weak metric tensor}_{\mu\nu} = \texttip{\texttip{h}{weak metric tensor}_{{\mu\nu}}}{weak covariant metric tensor component}-\frac{1}{2}\texttip{\texttip{h}{weak metric tensor}^{\alpha}_{\phantom{\alpha}\alpha}}{weak metric tensor trace}\texttip{\texttip{\eta}{Minkoski metric tensor +---}_{\mu\nu}}{covariant Minkoski metric components +---}}}\texttip{{}^{,\nu}}{Partial derivative}=0. 
}\label{eq-dedonderTR}\end{equation}

In this form we can very easily recover the conservation equation for
the stress-energy tensor. We take the divergence of
\ref{eq-linearizedTR} and use \ref{eq-dedonderTR} to get:

\begin{equation}\protect\hypertarget{eq-SEcons}{}{
    \texttip{\texttip{T}{Stress-Energy tensor, or energy-momentum tensor}_{{\mu\nu}}}{covariant Stress-Energy tensor component}\texttip{{}^{,\nu}}{Partial derivative}=0.
}\label{eq-SEcons}\end{equation}

Let us look at what sorts of solutions come out of these equations.

\hypertarget{homogenous-solutions}{%
\section{Homogenous solutions}\label{homogenous-solutions}}

The simplest first step is to consider the homogeous solution, as all
solutions will involve these terms. Setting
\(\texttip{\texttip{T}{Stress-Energy tensor, or energy-momentum tensor}_{{\mu\nu}}}{covariant Stress-Energy tensor component}=0\)
or
\(\texttip{\texttip{S}{source tensor}_{\mu\nu}}{covariant source tensor component}=0\)
yields an easily recognisable wave equation.
\begin{equation}\protect\hypertarget{eq-homogenous}{}{
   \mathtip{\Box_{SR}}{\texttip{\texttip{\eta}{Minkoski metric tensor +---}^{{\mu\nu}}}{contravariant Minkoski metric components +---}\partial_\mu\partial_\nu}\texttip{\texttip{h}{weak metric tensor}_{{\mu\nu}}}{weak covariant metric tensor component}= 0 
}\label{eq-homogenous}\end{equation}

The de Donder gauge (\ref{eq-dedonder}) and the remaining gauge
freedom\sidenote{\footnotesize from changes in coordinates such as
  \(\texttip{x^{\mu}}{contravariant mathbftor component}\to\texttip{x^{\mu}}{contravariant mathbftor component}+\texttip{\xi^{\mu}}{contravariant mathbftor component}\)
  with
  \(\mathtip{\Box_{SR}}{\texttip{\texttip{\eta}{Minkoski metric tensor +---}^{{\mu\nu}}}{contravariant Minkoski metric components +---}\partial_\mu\partial_\nu}\texttip{\xi^{\mu}}{contravariant mathbftor component}=0\)}
restricts the possible forms of this solution to having only helicity
\(\pm2\) physically significant components (see Weinberg
(1972)\marginpar{\begin{footnotesize}\leavevmode\vadjust pre{\protect\hypertarget{ref-weinbergGravitationCosmologyPrinciples1972}{}}%
Weinberg, Steven. 1972. \emph{Gravitation and Cosmology: Principles and
Applications of the General Theory of Relativity}. {New York}: {Wiley}.\vspace{2mm}\par\end{footnotesize}}).
Let us look at its generic form. The metric
\(\texttip{\texttip{h}{weak metric tensor}_{{\mu\nu}}}{weak covariant metric tensor component}\)
ought to be real-valued, thus we seek real solutions of the form

\[
    \texttip{\texttip{h}{weak metric tensor}_{{\mu\nu}}}{weak covariant metric tensor component}= \texttip{\texttip{\varepsilon}{Polarization tensor}_{{\mu\nu}}}{covariant Polarization tensor component}\oldexp{\mathtip{{\mathring{\imath}}}{\text{Complex unit: } {\mathring{\imath}}^2 = -1}k\cdot x} + \texttip{\texttip{\varepsilon^\ast}{Conjugate Polarization tensor}_{{\mu\nu}}}{covariant conjugate Polarization tensor component}\oldexp{-\mathtip{{\mathring{\imath}}}{\text{Complex unit: } {\mathring{\imath}}^2 = -1}k\cdot x},
\]

where
\(\texttip{\texttip{\varepsilon}{Polarization tensor}_{{\mu\nu}}}{covariant Polarization tensor component}\)
is the polarization tensor and \(k\) is the wave vector. The
polarization tensor is a symmetric rank-2 tensor, since
\(\texttip{\texttip{h}{weak metric tensor}_{{\mu\nu}}}{weak covariant metric tensor component}\)
is. Additionally we define: \[
    k\cdot x\equiv \texttip{\texttip{\eta}{Minkoski metric tensor +---}_{{\mu\nu}}}{covariant Minkoski metric components +---}\texttip{k^{\mu}}{contravariant mathbftor component} \texttip{k^{\nu}}{contravariant mathbftor component} = \texttip{k_{\mu}}{covariant mathbftor component} \texttip{x^{\mu}}{contravariant mathbftor component}.
\] Substituting into
\(\mathtip{\Box_{SR}}{\texttip{\texttip{\eta}{Minkoski metric tensor +---}^{{\mu\nu}}}{contravariant Minkoski metric components +---}\partial_\mu\partial_\nu}\texttip{\texttip{h}{weak metric tensor}_{{\mu\nu}}}{weak covariant metric tensor component}=0\)
gives
\(\texttip{k_{\mu}}{covariant mathbftor component} \texttip{k^{\mu}}{contravariant mathbftor component} \equiv k^2=0\)\sidenote{\footnotesize assuming
  non-zero perturbation
  \(\texttip{\texttip{h}{weak metric tensor}_{{\mu\nu}}}{weak covariant metric tensor component}\)
  of course} \sidenote{\footnotesize this is saying that the wavevector for the wave
  is null, thus that the wave propagates at the speed of light}. From
\ref{eq-dedonder} we have
\begin{equation}\protect\hypertarget{eq-polk}{}{
    \texttip{\texttip{\varepsilon}{Polarization tensor}^{\mu}_{\phantom{\mu}\nu}}{mixed Polarization tensor component}\texttip{k_{\mu}}{covariant mathbftor component} = \frac{1}{2}\texttip{\texttip{\varepsilon}{Polarization tensor}^{\alpha}{}_{\alpha}}{Polarization tensor trace}k_\nu.
}\label{eq-polk}\end{equation} As said at the beginning of this
subsection, we still have some remaining gauge freedom, which we now
fix, choosing the following coordinate change
\(\texttip{x^{\mu}}{contravariant mathbftor component}\to\texttip{x^{\mu}}{contravariant mathbftor component}+\texttip{\zeta^{\mu}}{contravariant mathbftor component}\)
where: \[
\texttip{\zeta^{\mu}}{contravariant mathbftor component}=\mathtip{{\mathring{\imath}}}{\text{Complex unit: } {\mathring{\imath}}^2 = -1}\texttip{A^{\mu}}{contravariant mathbftor component} \oldexp{\mathtip{{\mathring{\imath}}}{\text{Complex unit: } {\mathring{\imath}}^2 = -1}k\cdot x}    =  - \mathtip{{\mathring{\imath}}}{\text{Complex unit: } {\mathring{\imath}}^2 = -1}\texttip{\texttip{A^\ast}{conjugate of A}^{\mu}}{contravariant mathbftor component} \oldexp{ -\mathtip{{\mathring{\imath}}}{\text{Complex unit: } {\mathring{\imath}}^2 = -1}k\cdot x}
\]

Imposing this change yields the following modified perturbation:

\[
\texttip{h}{weak metric tensor}'_{\mu\nu}=\texttip{\texttip{h}{weak metric tensor}_{{\mu\nu}}}{weak covariant metric tensor component}-\frac{\partial^{} \texttip{\zeta_{\mu}}{covariant mathbftor component}}{\partial\texttip{x^{\nu}}{contravariant mathbftor component}^{}}-\frac{\partial^{} \texttip{\zeta_{\nu}}{covariant mathbftor component}}{\partial\texttip{x^{\mu}}{contravariant mathbftor component}^{}}=\texttip{\varepsilon}{Polarization tensor}'_{\mu\nu} \oldexp{\mathtip{{\mathring{\imath}}}{\text{Complex unit: } {\mathring{\imath}}^2 = -1}k\cdot x} + \texttip{\varepsilon}{Polarization tensor}^{\prime\ast}{}_{\mu\nu} \oldexp{-\mathtip{{\mathring{\imath}}}{\text{Complex unit: } {\mathring{\imath}}^2 = -1}k\cdot x}.
\] with

\begin{equation}\protect\hypertarget{eq-polTprime}{}{
\texttip{\varepsilon}{Polarization tensor}'_{\mu\nu} = \texttip{\texttip{\varepsilon}{Polarization tensor}_{{\mu\nu}}}{covariant Polarization tensor component}+ \texttip{k_{\mu}}{covariant mathbftor component}\texttip{A_{\nu}}{covariant mathbftor component} + \texttip{k_{\nu}}{covariant mathbftor component}\texttip{A_{\mu}}{covariant mathbftor component}
}\label{eq-polTprime}\end{equation}

Equations \ref{eq-polk} and \ref{eq-polTprime} reduce the free
components in the polarization tensor to just two. Additionally these
equations can conspire to yield a traceless polarisation tensor
\(\texttip{\texttip{\varepsilon}{Polarization tensor}^{\alpha}{}_{\alpha}}{Polarization tensor trace}=0\),
with
\(\texttip{\texttip{\varepsilon}{Polarization tensor}_{0\mu}}{covariant Polarization tensor component}=0\)
(see Carroll
(2019)\marginpar{\begin{footnotesize}\leavevmode\vadjust pre{\protect\hypertarget{ref-carrollSpacetimeGeometryIntroduction2019}{}}%
Carroll, Sean M. 2019. {``Spacetime and {Geometry}: {An Introduction} to
{General Relativity}.''} \emph{Higher Education from Cambridge
University Press}.
https://www.cambridge.org/highereducation/books/spacetime-and-geometry/38EDABF9E2BADCE6FBCF2B22DC12BFFE;
{Cambridge University Press}.
\url{https://doi.org/10.1017/9781108770385}.\vspace{2mm}\par\end{footnotesize}}).
This then extends to the metric perturbation, which when imposed to be
traceless, becomes equal to its trace-reversed counterpart. This is the
so-called transverse traceless gauge:

\[
\texttip{\texttip{h}{weak metric tensor}_{0\mu}}{weak covariant metric tensor component}=0,\quad \texttip{\texttip{h}{weak metric tensor}^{\alpha}_{\phantom{\alpha}\alpha}}{weak metric tensor trace}=0, \quad \texttip{\texttip{h}{weak metric tensor}_{{\mu\nu}}}{weak covariant metric tensor component}\texttip{{}^{,\nu}}{Partial derivative}=0.
\]

The metric perturbation in this gauge is written as:
\(\mathtip{\texttip{h^{TT}}{transverse traceless weak metric tensor}_{ij}}{{\text{transverse traceless weak covariant metric tensor component: } \texttip{h^{TT}}{transverse traceless weak metric tensor}_{0\mu}=0,\quad \texttip{h^{TT}}{transverse traceless weak metric tensor}^{i}_{\phantom{i}i}=0, \quad \texttip{h^{TT}}{transverse traceless weak metric tensor}_{ij}\texttip{{}^{,j}}{Partial derivative}=0.}}\).

\hypertarget{inhomogenous-solutions}{%
\section{Inhomogenous solutions}\label{inhomogenous-solutions}}

With the homogenous part of \ref{eq-linearized} accounted for, we can
now look at the inhomogenous part. The solution in the presence of a
source term of \ref{eq-linearized} will be heavily inspired by the
analogous problem in electromagnetism. If we define the following
retarded Green's function \[
    \texttip{\mathcal{G}_{\mathrm{ret}}\left(\texttip{x^{\mu}}{contravariant mathbftor component}-\texttip{x'^{\mu}}{contravariant mathbftor component}\right)}{ret Green's function} = -2\pi\,\texttip{\delta^{((4))}\left((\texttip{x^{\mu}}{contravariant mathbftor component}-\texttip{x'^{\mu}}{contravariant mathbftor component})^2\right)}{Dirac delta function}\texttip{\Theta^{()}\left(x^0>x^{\prime 0}\right)}{Heaviside step function},
\] which satisfies,

\[
    \mathtip{\Box_{SR}}{\texttip{\texttip{\eta}{Minkoski metric tensor +---}^{{\mu\nu}}}{contravariant Minkoski metric components +---}\partial_\mu\partial_\nu}\texttip{\mathcal{G}_{\mathrm{ret}}\left(\texttip{x^{\mu}}{contravariant mathbftor component}-\texttip{x'^{\mu}}{contravariant mathbftor component}\right)}{ret Green's function} = \texttip{\delta^{((4))}\left(\texttip{x^{\mu}}{contravariant mathbftor component}-\texttip{x'^{\mu}}{contravariant mathbftor component}\right)}{Dirac delta function}.
\]

Then, the solution to

\begin{equation}\protect\hypertarget{eq-EFEwave}{}{
 \mathtip{\Box_{SR}}{\texttip{\texttip{\eta}{Minkoski metric tensor +---}^{{\mu\nu}}}{contravariant Minkoski metric components +---}\partial_\mu\partial_\nu}\texttip{\texttip{h}{weak metric tensor}_{{\mu\nu}}}{weak covariant metric tensor component}=  -16 \pi \oldG\texttip{\texttip{S}{source tensor}_{\mu\nu}}{covariant source tensor component}
}\label{eq-EFEwave}\end{equation}

is given by:

\begin{equation}\protect\hypertarget{eq-hsolInhom}{}{
    \texttip{\texttip{h}{weak metric tensor}_{{\mu\nu}}}{weak covariant metric tensor component}(x) = 8G\int\,\mathrm{d}^{4}x'\,\,\texttip{\mathcal{G}_{\mathrm{ret}}\left(\texttip{x^{\mu}}{contravariant mathbftor component}-\texttip{x'^{\mu}}{contravariant mathbftor component}\right)}{ret Green's function}\texttip{\texttip{S}{source tensor}_{\mu\nu}}{covariant source tensor component}(x'),
}\label{eq-hsolInhom}\end{equation}

\marginnote{\begin{footnotesize}

One gets the parallel solution to the trace reversed equation
\ref{eq-linearizedTR} by swapping \(\texttip{S}{source tensor}\) with
\(\texttip{T}{Stress-Energy tensor, or energy-momentum tensor}\) and all
\(\texttip{h}{weak metric tensor}\) with
\(\texttip{\bar{h}}{trace reversed weak metric tensor}\)

\end{footnotesize}}

since when we plug in \ref{eq-hsolInhom} into \ref{eq-EFEwave} \[
    \Box_x\texttip{\texttip{h}{weak metric tensor}_{{\mu\nu}}}{weak covariant metric tensor component}= 8\oldG\int\,\mathrm{d}^{4}x'\,\par[\big]{(\underbrace{\Box_x\texttip{\mathcal{G}_{\mathrm{ret}}\left(\texttip{x^{\mu}}{contravariant mathbftor component}-\texttip{x'^{\mu}}{contravariant mathbftor component}\right)}{ret Green's function}}_{=-2\pi\,\texttip{\delta^{()}\left(\texttip{x^{\mu}}{contravariant mathbftor component}-\texttip{x'^{\mu}}{contravariant mathbftor component}\right)}{Dirac delta function}}} \texttip{\texttip{S}{source tensor}_{\mu\nu}}{covariant source tensor component}(x') = -16\pi \oldG\texttip{\texttip{S}{source tensor}_{\mu\nu}}{covariant source tensor component}(x).
\] We can perform the \(x^{\prime0}=t'\) integration in
\ref{eq-hsolInhom}, with the delta function, setting
\(t'=t-\vert\texttip{x^{\mu}}{contravariant mathbftor component}-\texttip{x'^{\mu}}{contravariant mathbftor component} \vert= t_r\)
the retarded time, i.e: \[
\texttip{\texttip{h}{weak metric tensor}_{{\mu\nu}}}{weak covariant metric tensor component}(x) = 8\oldG\int\,\mathrm{d}^{3}\mathbf{y}\,\texttip{\texttip{S}{source tensor}_{\mu\nu}}{covariant source tensor component}(t,\mathbf{y})\,\mathrm{d}^{}t'\,\frac{\delta(t'-(t-\vert\mathbf{x}-\mathbf{y} \vert))}{2\vert\mathbf{x}-\mathbf{y} \vert} 
\]

\begin{equation}\protect\hypertarget{eq-solInhomt}{}{
\texttip{\texttip{h}{weak metric tensor}_{{\mu\nu}}}{weak covariant metric tensor component}(\mathbf{x},t) = 4\oldG\int\,\mathrm{d}^{3}\mathbf{y}\,\frac{\texttip{\texttip{S}{source tensor}_{\mu\nu}}{covariant source tensor component}(t-\vert\mathbf{x}-\mathbf{y} \vert,\mathbf{y})}{\vert\mathbf{x}-\mathbf{y} \vert}.
}\label{eq-solInhomt}\end{equation}

We can interpret the solution at \((\mathbf{x},t)\) above, as the
perturbation due to the sumed up contributions of all the sources at the
point \((\mathbf{x}-\mathbf{y},t_r)\) on the past light cone. Put
differently this will be the gravitational radiation produced by the
source
\(\texttip{\texttip{S}{source tensor}_{\mu\nu}}{covariant source tensor component}\).
Additionally, the form of the time argument of the source tensor,
imposed by the definition of the Green's function, shows that the
radiation propagates with velocity \(=1=c\). This retarded solution
satisfies the harmonic coordinate condition of \ref{eq-linearized}, as
required. Indeed, we have

\[
    \texttip{\texttip{T}{Stress-Energy tensor, or energy-momentum tensor}_{{\mu\nu}}}{covariant Stress-Energy tensor component}\texttip{{}_{;\mu}}{Covariant derivative}=0\ \Rightarrow\ \texttip{\texttip{T}{Stress-Energy tensor, or energy-momentum tensor}_{{\mu\nu}}}{covariant Stress-Energy tensor component}\texttip{{}_{,\mu}}{Partial derivative}+\underbracket{\Gamma\,\Gamma\,}_{\mathclap{\mathrm{non-linear}}}=0\ \Rightarrow\ \texttip{\texttip{T}{Stress-Energy tensor, or energy-momentum tensor}_{{\mu\nu}}}{covariant Stress-Energy tensor component}\texttip{{}_{,\mu}}{Partial derivative}=0,
\]

ignoring non-linearities. Then,

\[
    \texttip{\texttip{S}{source tensor}^{\mu\nu}}{contravariant source tensor component}\texttip{{}_{,\mu}}{Partial derivative} = \partial_\mu\left(\texttip{\texttip{T}{Stress-Energy tensor, or energy-momentum tensor}_{{\mu\nu}}}{covariant Stress-Energy tensor component}-\frac{\texttip{\texttip{\eta}{Minkoski metric tensor +---}^{\mu\nu}}{contravariant Minkoski metric components +---}}{2}\texttip{\texttip{T}{Stress-Energy tensor, or energy-momentum tensor}^{\alpha}{}_{\alpha}}{Stress-Energy tensor trace}\right) = -\frac{\texttip{\texttip{\eta}{Minkoski metric tensor +---}^{\mu\nu}}{contravariant Minkoski metric components +---}}{2}\texttip{\texttip{T}{Stress-Energy tensor, or energy-momentum tensor}^{\alpha}{}_{\alpha}}{Stress-Energy tensor trace}\texttip{{}_{,\mu}}{Partial derivative}.
\]

Also

\[
    \texttip{\texttip{S}{source tensor}^{\mu}{}_{\mu}}{source tensor trace} = \texttip{\texttip{T}{Stress-Energy tensor, or energy-momentum tensor}^{\mu}{}_{\mu}}{Stress-Energy tensor trace} -\frac{\texttip{\delta_\mu^\mu}{Kronecker delta}}{2}\texttip{\texttip{T}{Stress-Energy tensor, or energy-momentum tensor}^{\alpha}{}_{\alpha}}{Stress-Energy tensor trace}= -\texttip{\texttip{T}{Stress-Energy tensor, or energy-momentum tensor}^{\alpha}{}_{\alpha}}{Stress-Energy tensor trace}.
\]

Thus

\[
\begin{split}
    \texttip{\texttip{S}{source tensor}^{\mu\nu}}{contravariant source tensor component}\texttip{{}_{,\mu}}{Partial derivative} &= \frac{1}{2}\texttip{\texttip{S}{source tensor}^{\alpha}{}_{\alpha}}{source tensor trace}\texttip{{}_{,\mu}}{Partial derivative}\texttip{\texttip{\eta}{Minkoski metric tensor +---}^{\mu\nu}}{contravariant Minkoski metric components +---} \\
    \texttip{\texttip{S}{source tensor}^{\mu}_{\phantom{\mu}\nu}}{mixed source tensor component}\texttip{{}_{,\mu}}{Partial derivative} &= \frac{1}{2}\texttip{\texttip{S}{source tensor}^{\alpha}{}_{\alpha}}{source tensor trace}\texttip{{}_{,\nu}}{Partial derivative}.
\end{split}
\]

Then

\[
\begin{split}
    \texttip{\texttip{h}{weak metric tensor}^{\mu}_{\phantom{\mu}\nu}}{weak mixed metric tensor component}\texttip{{}_{,\mu}}{Partial derivative} &= 8\oldG\frac{\partial^{} }{\partial\texttip{x^{\mu}}{contravariant mathbftor component}^{}} \int\,\mathrm{d}^{4}x'\, \texttip{\mathcal{G}_{\mathrm{ret}}\left(\texttip{x^{\mu}}{contravariant mathbftor component}-\texttip{x'^{\mu}}{contravariant mathbftor component}\right)}{ret Green's function} \texttip{\texttip{S}{source tensor}^{\mu}_{\phantom{\mu}\nu}}{mixed source tensor component}(x') \\
&= 8\oldG\int\,\mathrm{d}^{4}x'\,\,\frac{\partial^{} \texttip{\mathcal{G}_{\mathrm{ret}}\left(\texttip{x^{\mu}}{contravariant mathbftor component}-\texttip{x'^{\mu}}{contravariant mathbftor component}\right)}{ret Green's function}}{\partial \texttip{x^{\mu}}{contravariant mathbftor component}^{}} \texttip{\texttip{S}{source tensor}^{\mu}_{\phantom{\mu}\nu}}{mixed source tensor component}(x') \\
&= -8\oldG\int\,\mathrm{d}^{4}x'\,\frac{\partial^{} \texttip{\mathcal{G}_{\mathrm{ret}}\left(\texttip{x^{\mu}}{contravariant mathbftor component}-\texttip{x'^{\mu}}{contravariant mathbftor component}\right)}{ret Green's function}}{\partial\texttip{x'^{\mu}}{contravariant mathbftor component}^{}}\texttip{\texttip{S}{source tensor}^{\mu}_{\phantom{\mu}\nu}}{mixed source tensor component}(x') \\
    &= \underbrace{8\oldG\texttip{\mathcal{G}_{\mathrm{ret}}\left(\texttip{x^{\mu}}{contravariant mathbftor component}-\texttip{x'^{\mu}}{contravariant mathbftor component}\right)}{ret Green's function}\texttip{\delta_{\nu}^{\mu}}{Kronecker delta}(x')}_{=0\ \mathrm{for}\ x=x'} + 8G\int\,\mathrm{d}^{4}x'\,\,\texttip{\mathcal{G}_{\mathrm{ret}}\left(\texttip{x^{\mu}}{contravariant mathbftor component}-\texttip{x'^{\mu}}{contravariant mathbftor component}\right)}{ret Green's function}\frac{\partial^{} \texttip{\texttip{S}{source tensor}^{\mu}_{\phantom{\mu}\nu}}{mixed source tensor component}(x')}{\partial\texttip{x'^{\mu}}{contravariant mathbftor component}^{}} \\
    &= 8G\int\,\mathrm{d}^{4}x'\,\,\texttip{\mathcal{G}_{\mathrm{ret}}\left(\texttip{x^{\mu}}{contravariant mathbftor component}-\texttip{x'^{\mu}}{contravariant mathbftor component}\right)}{ret Green's function} \frac{1}{2}\frac{\partial^{} \texttip{\texttip{S}{source tensor}^{\alpha}{}_{\alpha}}{source tensor trace}(x')}{\partial\texttip{x'^{\mu}}{contravariant mathbftor component}^{}} \\
    &=\quad \dots \quad \text{repeat in reverse} \\
    &= \frac{\partial^{} }{\partial\texttip{x^{\mu}}{contravariant mathbftor component}^{}} \Big[8\oldG\int\,\mathrm{d}^{4}x'\,\,\texttip{\mathcal{G}_{\mathrm{ret}}\left(\texttip{x^{\mu}}{contravariant mathbftor component}-\texttip{x'^{\mu}}{contravariant mathbftor component}\right)}{ret Green's function} \frac{1}{2}\texttip{\texttip{S}{source tensor}^{\alpha}{}_{\alpha}}{source tensor trace}(x') \Big] \\
    &= \frac{1}{2} \texttip{\texttip{h}{weak metric tensor}^{\alpha}_{\phantom{\alpha}\alpha}}{weak metric tensor trace}\texttip{{}_{,\mu}}{Partial derivative}.\checkmark
\end{split}
\]

\hypertarget{gravitational-wave-sources}{%
\section{Gravitational Wave Sources}\label{gravitational-wave-sources}}

Now that we have the general form of the solutions to the linearized
Einstein equations, we can proceed to the analysis of the sources of
gravitational waves. The first step is to analyse the equations in the
frequency domain. We will use the following notation:

\[\mathtip{\mathcal{F}_{t}\left[\phi\right]}{\text{Fourier transform of $\phi$ in variable $t$: }{\mathcal{F}_{t}\left[\phi\right] = \int \,\mathrm{d}^{}t\, \phi(t) \oldexp{\mathtip{{\mathring{\imath}}}{\text{Complex unit: } {\mathring{\imath}}^2 = -1}\omega \cdot t}   }}(\omega,\mathbf{x})=\int \,\mathrm{d}^{}t\, \phi(t,\mathbf{x}) \oldexp{\mathtip{{\mathring{\imath}}}{\text{Complex unit: } {\mathring{\imath}}^2 = -1}\omega t}\]
\[\mathtip{\mathcal{F}^{-1}_{\omega}\left[\phi\right]}{\text{inverse Fourier transform of $\phi$ in variable $\omega$: }{\mathcal{F}_{\omega}\left[\phi\right] = \frac{1}{2\pi}\int  \,\mathrm{d}^{}\omega\,\oldexp{-\mathtip{{\mathring{\imath}}}{\text{Complex unit: } {\mathring{\imath}}^2 = -1}t \cdot \omega} \phi(\omega)  }}(t,\mathbf{x})=\int \frac{\,\mathrm{d}^{}\omega\,}{2\pi} \mathtip{\mathcal{F}_{t}\left[\phi\right]}{\text{Fourier transform of $\phi$ in variable $t$: }{\mathcal{F}_{t}\left[\phi\right] = \int \,\mathrm{d}^{}t\, \phi(t) \oldexp{\mathtip{{\mathring{\imath}}}{\text{Complex unit: } {\mathring{\imath}}^2 = -1}\omega \cdot t}   }}(\omega,\mathbf{x}) \oldexp{- \mathtip{{\mathring{\imath}}}{\text{Complex unit: } {\mathring{\imath}}^2 = -1}\omega t}\]

Let us look at the trace reversed solution, as the conservation equation
for
\begin{equation}\protect\hypertarget{eq-FTweakmet}{}{\begin{aligned} 
\mathtip{\mathcal{F}_{t}\left[\texttip{\texttip{h}{weak metric tensor}_{{\mu\nu}}}{weak covariant metric tensor component}\right]}{\text{Fourier transform of $\texttip{\texttip{h}{weak metric tensor}_{{\mu\nu}}}{weak covariant metric tensor component}$ in variable $t$: }{\mathcal{F}_{t}\left[\texttip{\texttip{h}{weak metric tensor}_{{\mu\nu}}}{weak covariant metric tensor component}\right] = \int \,\mathrm{d}^{}t\, \texttip{\texttip{h}{weak metric tensor}_{{\mu\nu}}}{weak covariant metric tensor component}(t) \oldexp{\mathtip{{\mathring{\imath}}}{\text{Complex unit: } {\mathring{\imath}}^2 = -1}\omega \cdot t}   }}(\omega, \mathbf{x}) &=\int \,\mathrm{d}^{}t\, \texttip{\texttip{h}{weak metric tensor}_{{\mu\nu}}}{weak covariant metric tensor component}(t,\mathbf{x}) \oldexp{\mathtip{{\mathring{\imath}}}{\text{Complex unit: } {\mathring{\imath}}^2 = -1}\omega t}
\\ &=4\oldG\int\,\mathrm{d}^{3}\mathbf{y}\,\,\mathrm{d}^{}t\, \frac{\texttip{\texttip{S}{source tensor}_{\mu\nu}}{covariant source tensor component}(t-\vert\mathbf{x}-\mathbf{y} \vert,\mathbf{y})}{\vert\mathbf{x}-\mathbf{y} \vert}  \oldexp{\mathtip{{\mathring{\imath}}}{\text{Complex unit: } {\mathring{\imath}}^2 = -1}\omega t}
\\ &=4\oldG\int\,\mathrm{d}^{3}\mathbf{y}\,\,\mathrm{d}^{}t_r\, \frac{\texttip{\texttip{S}{source tensor}_{\mu\nu}}{covariant source tensor component}(t_r,\mathbf{y})}{\vert\mathbf{x}-\mathbf{y} \vert}  \oldexp{\mathtip{{\mathring{\imath}}}{\text{Complex unit: } {\mathring{\imath}}^2 = -1}\omega t_r}\oldexp{\mathtip{{\mathring{\imath}}}{\text{Complex unit: } {\mathring{\imath}}^2 = -1}\omega\vert\mathbf{x}-\mathbf{y} \vert}
\\ &=4\oldG\int\,\mathrm{d}^{3}\mathbf{y}\, \frac{\mathtip{\mathcal{F}_{t_r}\left[\texttip{\texttip{S}{source tensor}_{\mu\nu}}{covariant source tensor component}\right]}{\text{Fourier transform of $\texttip{\texttip{S}{source tensor}_{\mu\nu}}{covariant source tensor component}$ in variable $t_r$: }{\mathcal{F}_{t_r}\left[\texttip{\texttip{S}{source tensor}_{\mu\nu}}{covariant source tensor component}\right] = \int \,\mathrm{d}^{}t_r\, \texttip{\texttip{S}{source tensor}_{\mu\nu}}{covariant source tensor component}(t_r) \oldexp{\mathtip{{\mathring{\imath}}}{\text{Complex unit: } {\mathring{\imath}}^2 = -1}\omega \cdot t_r}   }}(\omega,\mathbf{y})}{\vert\mathbf{x}-\mathbf{y} \vert} \oldexp{\mathtip{{\mathring{\imath}}}{\text{Complex unit: } {\mathring{\imath}}^2 = -1}\omega\vert\mathbf{x}-\mathbf{y} \vert} 
\end{aligned}}\label{eq-FTweakmet}\end{equation}

We can now apply various approximations to this form of the
perturbation. The first is to consider that we look at the radiation
only in the so called \emph{wave zone}, at a distance
\(r=\vert\mathbf{x} \vert\) much larger than the dimensions of the
source \(R=\vert\mathbf{y} \vert_\text{max}\). Additionally we assume
that \(r \gg \frac{1}{\omega}\), i.e long wavelengths don't dominate.
Finally we assume that \(r \gg \omega R^2\), i.e.~the ratio of \(R\) to
the wavelength is not comparable to the ratio of \(r\) to \(R\). Using
this approximation we can write:

\[
\begin{split}
    \vert\mathbf{x}-\mathbf{y} \vert &= r\left(1-2\hat{\mathbf{x}}\cdot\mathbf{y}+\frac{\mathbf{x}^{\prime2}}{r^2}\right)^{1/2} \\
    \vert\mathbf{x}-\mathbf{y} \vert &\approx r\left(1-\frac{\hat{\mathbf{x}}\cdot\mathbf{y}}{r}\right) \quad \text{with: } \hat{\mathbf{x}} = \frac{\mathbf{x}}{r}.
\end{split}
\]

If we additionally further separate the scales in the following
way\sidenote{\footnotesize we just add the condition that \(\frac{1}{\omega}\gg R\),
  that is the source radius is much smaller than the wavelength}

\[
r\gg \{\frac{1}{\omega},\omega R^2 \}\gg R,
\]

Then \ref{eq-FTweakmet} becomes much simpler\sidenote{\footnotesize the approximations
  all conspire to be able to neglect the \(\mathbf{y}\) dependence of
  \(\oldexp{\mathtip{{\mathring{\imath}}}{\text{Complex unit: } {\mathring{\imath}}^2 = -1}\omega\vert\mathbf{x}-\mathbf{y} \vert}\)
  in the integral}:

\begin{equation}\protect\hypertarget{eq-FTweakmet2}{}{
\mathtip{\mathcal{F}_{t}\left[\texttip{\texttip{h}{weak metric tensor}_{{\mu\nu}}}{weak covariant metric tensor component}\right]}{\text{Fourier transform of $\texttip{\texttip{h}{weak metric tensor}_{{\mu\nu}}}{weak covariant metric tensor component}$ in variable $t$: }{\mathcal{F}_{t}\left[\texttip{\texttip{h}{weak metric tensor}_{{\mu\nu}}}{weak covariant metric tensor component}\right] = \int \,\mathrm{d}^{}t\, \texttip{\texttip{h}{weak metric tensor}_{{\mu\nu}}}{weak covariant metric tensor component}(t) \oldexp{\mathtip{{\mathring{\imath}}}{\text{Complex unit: } {\mathring{\imath}}^2 = -1}\omega \cdot t}   }}(\omega, \mathbf{x})=4\oldG\frac{\oldexp{\mathtip{{\mathring{\imath}}}{\text{Complex unit: } {\mathring{\imath}}^2 = -1}\omega R}}{R} \int\,\mathrm{d}^{3}\mathbf{y}\, \mathtip{\mathcal{F}_{t}\left[\texttip{\texttip{S}{source tensor}_{\mu\nu}}{covariant source tensor component}\right]}{\text{Fourier transform of $\texttip{\texttip{S}{source tensor}_{\mu\nu}}{covariant source tensor component}$ in variable $t$: }{\mathcal{F}_{t}\left[\texttip{\texttip{S}{source tensor}_{\mu\nu}}{covariant source tensor component}\right] = \int \,\mathrm{d}^{}t\, \texttip{\texttip{S}{source tensor}_{\mu\nu}}{covariant source tensor component}(t) \oldexp{\mathtip{{\mathring{\imath}}}{\text{Complex unit: } {\mathring{\imath}}^2 = -1}\omega \cdot t}   }}(\omega,\mathbf{y}) 
}\label{eq-FTweakmet2}\end{equation}

Now let us look at the fourier transform of the source term. By
definition we have

\[
\mathtip{\mathcal{F}_{t}\left[\texttip{\texttip{S}{source tensor}_{\mu\nu}}{covariant source tensor component}\right]}{\text{Fourier transform of $\texttip{\texttip{S}{source tensor}_{\mu\nu}}{covariant source tensor component}$ in variable $t$: }{\mathcal{F}_{t}\left[\texttip{\texttip{S}{source tensor}_{\mu\nu}}{covariant source tensor component}\right] = \int \,\mathrm{d}^{}t\, \texttip{\texttip{S}{source tensor}_{\mu\nu}}{covariant source tensor component}(t) \oldexp{\mathtip{{\mathring{\imath}}}{\text{Complex unit: } {\mathring{\imath}}^2 = -1}\omega \cdot t}   }}(\omega,\mathbf{y})=\mathtip{\mathcal{F}_{t}\left[\texttip{\texttip{T}{Stress-Energy tensor, or energy-momentum tensor}_{{\mu\nu}}}{covariant Stress-Energy tensor component}\right]}{\text{Fourier transform of $\texttip{\texttip{T}{Stress-Energy tensor, or energy-momentum tensor}_{{\mu\nu}}}{covariant Stress-Energy tensor component}$ in variable $t$: }{\mathcal{F}_{t}\left[\texttip{\texttip{T}{Stress-Energy tensor, or energy-momentum tensor}_{{\mu\nu}}}{covariant Stress-Energy tensor component}\right] = \int \,\mathrm{d}^{}t\, \texttip{\texttip{T}{Stress-Energy tensor, or energy-momentum tensor}_{{\mu\nu}}}{covariant Stress-Energy tensor component}(t) \oldexp{\mathtip{{\mathring{\imath}}}{\text{Complex unit: } {\mathring{\imath}}^2 = -1}\omega \cdot t}   }}(\omega,\mathbf{y})+\tfrac{1}{2}\texttip{\texttip{\eta}{Minkoski metric tensor +---}_{{\mu\nu}}}{covariant Minkoski metric components +---}\mathtip{\mathcal{F}_{t}\left[\texttip{\texttip{T}{Stress-Energy tensor, or energy-momentum tensor}^{\alpha}{}_{\alpha}}{Stress-Energy tensor trace}\right]}{\text{Fourier transform of $\texttip{\texttip{T}{Stress-Energy tensor, or energy-momentum tensor}^{\alpha}{}_{\alpha}}{Stress-Energy tensor trace}$ in variable $t$: }{\mathcal{F}_{t}\left[\texttip{\texttip{T}{Stress-Energy tensor, or energy-momentum tensor}^{\alpha}{}_{\alpha}}{Stress-Energy tensor trace}\right] = \int \,\mathrm{d}^{}t\, \texttip{\texttip{T}{Stress-Energy tensor, or energy-momentum tensor}^{\alpha}{}_{\alpha}}{Stress-Energy tensor trace}(t) \oldexp{\mathtip{{\mathring{\imath}}}{\text{Complex unit: } {\mathring{\imath}}^2 = -1}\omega \cdot t}   }}(\omega,\mathbf{y})
\]

Thus the term to analyse is actually
\(\mathtip{\mathcal{F}_{t}\left[\texttip{\texttip{T}{Stress-Energy tensor, or energy-momentum tensor}_{{\mu\nu}}}{covariant Stress-Energy tensor component}\right]}{\text{Fourier transform of $\texttip{\texttip{T}{Stress-Energy tensor, or energy-momentum tensor}_{{\mu\nu}}}{covariant Stress-Energy tensor component}$ in variable $t$: }{\mathcal{F}_{t}\left[\texttip{\texttip{T}{Stress-Energy tensor, or energy-momentum tensor}_{{\mu\nu}}}{covariant Stress-Energy tensor component}\right] = \int \,\mathrm{d}^{}t\, \texttip{\texttip{T}{Stress-Energy tensor, or energy-momentum tensor}_{{\mu\nu}}}{covariant Stress-Energy tensor component}(t) \oldexp{\mathtip{{\mathring{\imath}}}{\text{Complex unit: } {\mathring{\imath}}^2 = -1}\omega \cdot t}   }}\).
We can use the conservation equation \ref{eq-SEcons} in fourrier
\(t\)-space to write:

\begin{equation}\protect\hypertarget{eq-FTSEcons}{}{
- \mathtip{\mathcal{F}_{t}\left[\texttip{\texttip{T}{Stress-Energy tensor, or energy-momentum tensor}_{i\mu}}{covariant Stress-Energy tensor component}\right]}{\text{Fourier transform of $\texttip{\texttip{T}{Stress-Energy tensor, or energy-momentum tensor}_{i\mu}}{covariant Stress-Energy tensor component}$ in variable $t$: }{\mathcal{F}_{t}\left[\texttip{\texttip{T}{Stress-Energy tensor, or energy-momentum tensor}_{i\mu}}{covariant Stress-Energy tensor component}\right] = \int \,\mathrm{d}^{}t\, \texttip{\texttip{T}{Stress-Energy tensor, or energy-momentum tensor}_{i\mu}}{covariant Stress-Energy tensor component}(t) \oldexp{\mathtip{{\mathring{\imath}}}{\text{Complex unit: } {\mathring{\imath}}^2 = -1}\omega \cdot t}   }}\texttip{{}^{,i}}{Partial derivative} = \mathtip{{\mathring{\imath}}}{\text{Complex unit: } {\mathring{\imath}}^2 = -1}\omega\mathtip{\mathcal{F}_{t}\left[\texttip{\texttip{T}{Stress-Energy tensor, or energy-momentum tensor}_{0\mu}}{covariant Stress-Energy tensor component}\right]}{\text{Fourier transform of $\texttip{\texttip{T}{Stress-Energy tensor, or energy-momentum tensor}_{0\mu}}{covariant Stress-Energy tensor component}$ in variable $t$: }{\mathcal{F}_{t}\left[\texttip{\texttip{T}{Stress-Energy tensor, or energy-momentum tensor}_{0\mu}}{covariant Stress-Energy tensor component}\right] = \int \,\mathrm{d}^{}t\, \texttip{\texttip{T}{Stress-Energy tensor, or energy-momentum tensor}_{0\mu}}{covariant Stress-Energy tensor component}(t) \oldexp{\mathtip{{\mathring{\imath}}}{\text{Complex unit: } {\mathring{\imath}}^2 = -1}\omega \cdot t}   }}
}\label{eq-FTSEcons}\end{equation}

This equation becomes algebraic when we further fourrier transorm in
\(\mathbf{x}\)-space:

\[
\mathtip{\hat{\texttip{T}{Stress-Energy tensor, or energy-momentum tensor}}}{\text{Fourier transform of $\texttip{T}{Stress-Energy tensor, or energy-momentum tensor}$ in variable $\mathbf{x}$: }{\hat{\texttip{T}{Stress-Energy tensor, or energy-momentum tensor}}\left(\oldk\right)
 = \int \,\mathrm{d}^{}\mathbf{x}\, \texttip{T}{Stress-Energy tensor, or energy-momentum tensor}(\mathbf{x}) \oldexp{\mathtip{{\mathring{\imath}}}{\text{Complex unit: } {\mathring{\imath}}^2 = -1}\oldk \cdot \mathbf{x}}}}_{\mu\nu}(\texttip{k^{\alpha}}{contravariant mathbftor component}) =\mathtip{\hat{\texttip{T}{Stress-Energy tensor, or energy-momentum tensor}}}{\text{Fourier transform of $\texttip{T}{Stress-Energy tensor, or energy-momentum tensor}$ in variable $\mathbf{x}$: }{\hat{\texttip{T}{Stress-Energy tensor, or energy-momentum tensor}}\left(\oldk\right)
 = \int \,\mathrm{d}^{}\mathbf{x}\, \texttip{T}{Stress-Energy tensor, or energy-momentum tensor}(\mathbf{x}) \oldexp{\mathtip{{\mathring{\imath}}}{\text{Complex unit: } {\mathring{\imath}}^2 = -1}\oldk \cdot \mathbf{x}}}}_{\mu\nu}(\omega,\oldk) = \int \,\mathrm{d}^{3}\mathbf{y}\, \mathtip{\mathcal{F}_{t}\left[\texttip{\texttip{T}{Stress-Energy tensor, or energy-momentum tensor}_{{\mu\nu}}}{covariant Stress-Energy tensor component}\right]}{\text{Fourier transform of $\texttip{\texttip{T}{Stress-Energy tensor, or energy-momentum tensor}_{{\mu\nu}}}{covariant Stress-Energy tensor component}$ in variable $t$: }{\mathcal{F}_{t}\left[\texttip{\texttip{T}{Stress-Energy tensor, or energy-momentum tensor}_{{\mu\nu}}}{covariant Stress-Energy tensor component}\right] = \int \,\mathrm{d}^{}t\, \texttip{\texttip{T}{Stress-Energy tensor, or energy-momentum tensor}_{{\mu\nu}}}{covariant Stress-Energy tensor component}(t) \oldexp{\mathtip{{\mathring{\imath}}}{\text{Complex unit: } {\mathring{\imath}}^2 = -1}\omega \cdot t}   }}(\omega,\mathbf{y}) \oldexp{\mathtip{{\mathring{\imath}}}{\text{Complex unit: } {\mathring{\imath}}^2 = -1}\oldk\cdot\mathbf{y}}
\] Then the conservation equation becomes: \[
\texttip{k^{\mu}}{contravariant mathbftor component}\texttip{\texttip{T}{Stress-Energy tensor, or energy-momentum tensor}_{{\mu\nu}}}{covariant Stress-Energy tensor component}(\omega,\oldk) = 0
\]

These four equations enable us to just care about the purely spacelike
components of
\(\texttip{\texttip{T}{Stress-Energy tensor, or energy-momentum tensor}_{{\mu\nu}}}{covariant Stress-Energy tensor component}\).
Let us apply \ref{eq-FTSEcons} to itself to obtain:

\[
\mathtip{\mathcal{F}_{t}\left[\texttip{\texttip{T}{Stress-Energy tensor, or energy-momentum tensor}_{ij}}{covariant Stress-Energy tensor component}\right]}{\text{Fourier transform of $\texttip{\texttip{T}{Stress-Energy tensor, or energy-momentum tensor}_{ij}}{covariant Stress-Energy tensor component}$ in variable $t$: }{\mathcal{F}_{t}\left[\texttip{\texttip{T}{Stress-Energy tensor, or energy-momentum tensor}_{ij}}{covariant Stress-Energy tensor component}\right] = \int \,\mathrm{d}^{}t\, \texttip{\texttip{T}{Stress-Energy tensor, or energy-momentum tensor}_{ij}}{covariant Stress-Energy tensor component}(t) \oldexp{\mathtip{{\mathring{\imath}}}{\text{Complex unit: } {\mathring{\imath}}^2 = -1}\omega \cdot t}   }}\texttip{{}^{,ij}}{Partial derivative}=-\omega^2\mathtip{\mathcal{F}_{t}\left[\texttip{\texttip{T}{Stress-Energy tensor, or energy-momentum tensor}_{00}}{covariant Stress-Energy tensor component}\right]}{\text{Fourier transform of $\texttip{\texttip{T}{Stress-Energy tensor, or energy-momentum tensor}_{00}}{covariant Stress-Energy tensor component}$ in variable $t$: }{\mathcal{F}_{t}\left[\texttip{\texttip{T}{Stress-Energy tensor, or energy-momentum tensor}_{00}}{covariant Stress-Energy tensor component}\right] = \int \,\mathrm{d}^{}t\, \texttip{\texttip{T}{Stress-Energy tensor, or energy-momentum tensor}_{00}}{covariant Stress-Energy tensor component}(t) \oldexp{\mathtip{{\mathring{\imath}}}{\text{Complex unit: } {\mathring{\imath}}^2 = -1}\omega \cdot t}   }}
\] which when multiplied by
\(\texttip{x_{m}}{covariant mathbftor component}\texttip{x_{n}}{covariant mathbftor component}\)
and integrated over \(\mathbf{x}\) gives\sidenote{\footnotesize two integrations by
  parts cancels the
  \(\texttip{x_{m}}{covariant mathbftor component}\texttip{x_{n}}{covariant mathbftor component}\)
  term in the {LHS}, since boundary terms are 0 (the source is finite)
  we have
  \[\int \,\mathrm{d}^{}x\, \texttip{x_{m}}{covariant mathbftor component}\texttip{x_{n}}{covariant mathbftor component} \mathtip{\mathcal{F}_{t}\left[\texttip{\texttip{T}{Stress-Energy tensor, or energy-momentum tensor}_{ij}}{covariant Stress-Energy tensor component}\right]}{\text{Fourier transform of $\texttip{\texttip{T}{Stress-Energy tensor, or energy-momentum tensor}_{ij}}{covariant Stress-Energy tensor component}$ in variable $t$: }{\mathcal{F}_{t}\left[\texttip{\texttip{T}{Stress-Energy tensor, or energy-momentum tensor}_{ij}}{covariant Stress-Energy tensor component}\right] = \int \,\mathrm{d}^{}t\, \texttip{\texttip{T}{Stress-Energy tensor, or energy-momentum tensor}_{ij}}{covariant Stress-Energy tensor component}(t) \oldexp{\mathtip{{\mathring{\imath}}}{\text{Complex unit: } {\mathring{\imath}}^2 = -1}\omega \cdot t}   }}\texttip{{}^{,ij}}{Partial derivative}=\int\,\mathrm{d}^{}x\, \texttip{x_{m}}{covariant mathbftor component}\texttip{x_{n}}{covariant mathbftor component}\texttip{{}^{,ij}}{Partial derivative} \mathtip{\mathcal{F}_{t}\left[\texttip{\texttip{T}{Stress-Energy tensor, or energy-momentum tensor}_{ij}}{covariant Stress-Energy tensor component}\right]}{\text{Fourier transform of $\texttip{\texttip{T}{Stress-Energy tensor, or energy-momentum tensor}_{ij}}{covariant Stress-Energy tensor component}$ in variable $t$: }{\mathcal{F}_{t}\left[\texttip{\texttip{T}{Stress-Energy tensor, or energy-momentum tensor}_{ij}}{covariant Stress-Energy tensor component}\right] = \int \,\mathrm{d}^{}t\, \texttip{\texttip{T}{Stress-Energy tensor, or energy-momentum tensor}_{ij}}{covariant Stress-Energy tensor component}(t) \oldexp{\mathtip{{\mathring{\imath}}}{\text{Complex unit: } {\mathring{\imath}}^2 = -1}\omega \cdot t}   }}\]
  The hessian of
  \(\texttip{x_{m}}{covariant mathbftor component}\texttip{x_{n}}{covariant mathbftor component}\)
  is
  \((\texttip{\delta^i_m}{Kronecker delta}+\texttip{\delta^i_n}{Kronecker delta})(\texttip{\delta^j_n}{Kronecker delta}+\texttip{\delta^j_m}{Kronecker delta})\),
  but since
  \(\texttip{\texttip{T}{Stress-Energy tensor, or energy-momentum tensor}_{{\mu\nu}}}{covariant Stress-Energy tensor component}\)
  is symmetric the integral is
  \[\int \,\mathrm{d}^{}\mathbf{y}\, 2\mathtip{\mathcal{F}_{t}\left[\texttip{\texttip{T}{Stress-Energy tensor, or energy-momentum tensor}_{mn}}{covariant Stress-Energy tensor component}\right]}{\text{Fourier transform of $\texttip{\texttip{T}{Stress-Energy tensor, or energy-momentum tensor}_{mn}}{covariant Stress-Energy tensor component}$ in variable $t$: }{\mathcal{F}_{t}\left[\texttip{\texttip{T}{Stress-Energy tensor, or energy-momentum tensor}_{mn}}{covariant Stress-Energy tensor component}\right] = \int \,\mathrm{d}^{}t\, \texttip{\texttip{T}{Stress-Energy tensor, or energy-momentum tensor}_{mn}}{covariant Stress-Energy tensor component}(t) \oldexp{\mathtip{{\mathring{\imath}}}{\text{Complex unit: } {\mathring{\imath}}^2 = -1}\omega \cdot t}   }}(\omega,\mathbf{y}) \]}:
\[
\int \,\mathrm{d}^{}\mathbf{x}\, \mathtip{\mathcal{F}_{t}\left[\texttip{\texttip{T}{Stress-Energy tensor, or energy-momentum tensor}_{mn}}{covariant Stress-Energy tensor component}\right]}{\text{Fourier transform of $\texttip{\texttip{T}{Stress-Energy tensor, or energy-momentum tensor}_{mn}}{covariant Stress-Energy tensor component}$ in variable $t$: }{\mathcal{F}_{t}\left[\texttip{\texttip{T}{Stress-Energy tensor, or energy-momentum tensor}_{mn}}{covariant Stress-Energy tensor component}\right] = \int \,\mathrm{d}^{}t\, \texttip{\texttip{T}{Stress-Energy tensor, or energy-momentum tensor}_{mn}}{covariant Stress-Energy tensor component}(t) \oldexp{\mathtip{{\mathring{\imath}}}{\text{Complex unit: } {\mathring{\imath}}^2 = -1}\omega \cdot t}   }}(\omega,\mathbf{x})=-\frac{\omega}{2} \int \,\mathrm{d}^{}\mathbf{x}\, \texttip{x_{m}}{covariant mathbftor component}\texttip{x_{n}}{covariant mathbftor component} \mathtip{\mathcal{F}_{t}\left[\texttip{\texttip{T}{Stress-Energy tensor, or energy-momentum tensor}_{00}}{covariant Stress-Energy tensor component}\right]}{\text{Fourier transform of $\texttip{\texttip{T}{Stress-Energy tensor, or energy-momentum tensor}_{00}}{covariant Stress-Energy tensor component}$ in variable $t$: }{\mathcal{F}_{t}\left[\texttip{\texttip{T}{Stress-Energy tensor, or energy-momentum tensor}_{00}}{covariant Stress-Energy tensor component}\right] = \int \,\mathrm{d}^{}t\, \texttip{\texttip{T}{Stress-Energy tensor, or energy-momentum tensor}_{00}}{covariant Stress-Energy tensor component}(t) \oldexp{\mathtip{{\mathring{\imath}}}{\text{Complex unit: } {\mathring{\imath}}^2 = -1}\omega \cdot t}   }}(\omega,\mathbf{x})
\]

Notice the last integral is in fact the fourrier transform of (a third
of) the quadrupole moment tensor of the energy denisty\sidenote{\footnotesize \[q_{mn}=3\int \texttip{x_{m}}{covariant mathbftor component}\texttip{x_{n}}{covariant mathbftor component} \texttip{\texttip{T}{Stress-Energy tensor, or energy-momentum tensor}_{00}}{covariant Stress-Energy tensor component}(\omega,\mathbf{x})\]}.
We call it
\(\mathtip{\hat{q}}{\text{Fourier transform of $q$ in variable $t$: }{\hat{q}\left(\omega\right)
 = \int \,\mathrm{d}^{}t\, q(t) \oldexp{\mathtip{{\mathring{\imath}}}{\text{Complex unit: } {\mathring{\imath}}^2 = -1}\omega \cdot t}}}_{mn}(\omega)\)
and we can finally rewrite \ref{eq-FTweakmet2} as:

\[
\mathtip{\mathcal{F}_{t}\left[\texttip{\texttip{h}{weak metric tensor}_{{\mu\nu}}}{weak covariant metric tensor component}\right]}{\text{Fourier transform of $\texttip{\texttip{h}{weak metric tensor}_{{\mu\nu}}}{weak covariant metric tensor component}$ in variable $t$: }{\mathcal{F}_{t}\left[\texttip{\texttip{h}{weak metric tensor}_{{\mu\nu}}}{weak covariant metric tensor component}\right] = \int \,\mathrm{d}^{}t\, \texttip{\texttip{h}{weak metric tensor}_{{\mu\nu}}}{weak covariant metric tensor component}(t) \oldexp{\mathtip{{\mathring{\imath}}}{\text{Complex unit: } {\mathring{\imath}}^2 = -1}\omega \cdot t}   }}(\omega, \mathbf{x})=-\frac{2\oldG\omega^2}{3}\frac{\oldexp{\mathtip{{\mathring{\imath}}}{\text{Complex unit: } {\mathring{\imath}}^2 = -1}\omega R}}{R} \par{\mathtip{\hat{q}}{\text{Fourier transform of $q$ in variable $t$: }{\hat{q}\left(\omega\right)
 = \int \,\mathrm{d}^{}t\, q(t) \oldexp{\mathtip{{\mathring{\imath}}}{\text{Complex unit: } {\mathring{\imath}}^2 = -1}\omega \cdot t}}}_{mn}(\omega)+\tfrac{1}{2}\texttip{\texttip{\eta}{Minkoski metric tensor +---}_{{\mu\nu}}}{covariant Minkoski metric components +---}\mathtip{\hat{q}}{\text{Fourier transform of $q$ in variable $t$: }{\hat{q}\left(\omega\right)
 = \int \,\mathrm{d}^{}t\, q(t) \oldexp{\mathtip{{\mathring{\imath}}}{\text{Complex unit: } {\mathring{\imath}}^2 = -1}\omega \cdot t}}}^n{}_{n}(\omega)}
\]

Going back \(t\)-space we have:

\[
\begin{aligned}
\texttip{\texttip{h}{weak metric tensor}_{{\mu\nu}}}{weak covariant metric tensor component}(t,\mathbf{x})&=-\frac{\oldG}{3\pi R}\int \,\mathrm{d}^{}\omega\, \oldexp{-\mathtip{{\mathring{\imath}}}{\text{Complex unit: } {\mathring{\imath}}^2 = -1}\omega(t-R)} \omega^2 \par{\mathtip{\hat{q}}{\text{Fourier transform of $q$ in variable $t$: }{\hat{q}\left(\omega\right)
 = \int \,\mathrm{d}^{}t\, q(t) \oldexp{\mathtip{{\mathring{\imath}}}{\text{Complex unit: } {\mathring{\imath}}^2 = -1}\omega \cdot t}}}_{mn}(\omega)+\tfrac{1}{2}\texttip{\texttip{\eta}{Minkoski metric tensor +---}_{{\mu\nu}}}{covariant Minkoski metric components +---}\mathtip{\hat{q}}{\text{Fourier transform of $q$ in variable $t$: }{\hat{q}\left(\omega\right)
 = \int \,\mathrm{d}^{}t\, q(t) \oldexp{\mathtip{{\mathring{\imath}}}{\text{Complex unit: } {\mathring{\imath}}^2 = -1}\omega \cdot t}}}^n{}_{n}(\omega)}\\
&=\frac{\oldG}{3\pi R}  \frac{\mathrm{d}^{2} }{\mathrm{d}t^{2}}\int \,\mathrm{d}^{}\omega\, \oldexp{-\mathtip{{\mathring{\imath}}}{\text{Complex unit: } {\mathring{\imath}}^2 = -1}\omega(t_r)} \par{\mathtip{\hat{q}}{\text{Fourier transform of $q$ in variable $t$: }{\hat{q}\left(\omega\right)
 = \int \,\mathrm{d}^{}t\, q(t) \oldexp{\mathtip{{\mathring{\imath}}}{\text{Complex unit: } {\mathring{\imath}}^2 = -1}\omega \cdot t}}}_{mn}(\omega)+\tfrac{1}{2}\texttip{\texttip{\eta}{Minkoski metric tensor +---}_{{\mu\nu}}}{covariant Minkoski metric components +---}\mathtip{\hat{q}}{\text{Fourier transform of $q$ in variable $t$: }{\hat{q}\left(\omega\right)
 = \int \,\mathrm{d}^{}t\, q(t) \oldexp{\mathtip{{\mathring{\imath}}}{\text{Complex unit: } {\mathring{\imath}}^2 = -1}\omega \cdot t}}}^n{}_{n}(\omega)}\\
&=\frac{2\oldG}{3R}\frac{\mathrm{d}^{2} }{\mathrm{d}t^{2}}\par{q_{mn}(t_r)+\tfrac{1}{2}\texttip{\texttip{\eta}{Minkoski metric tensor +---}_{{\mu\nu}}}{covariant Minkoski metric components +---}q^n{}_n(t_r)}
\end{aligned}
\]

This equation has a very nice physical interpretation. The gravitational
wave produced by a non-relativistic source is proportional to the second
derivative of the quadrupole moment of the its energy denisty at the
time where the past light cone of the observer intersects the source
(\(t_r\)). The nature of gravitational radiation is in stark contrast to
the leading electromagnetic contribution which is due the the change in
the \emph{dipole} moment of the charge density. The change of the dipole
moment can be attributed to the change in center of charge (for
Electromagnetism ()), or mass (for {GR}), and while a center of charge
is free to move around, the center of mass (of an isolated) is fixed by
the conservation of momentum. The quadrupole moment, on the other hand,
is sensitive to the shape of the source, which a gravitional system can
modify. Finally the quadrupole radiation is subleading when compared to
dipole radiation. Thus on top of the much smaller coupling constant,
gravitional radiation is also weakened by this fact, and thus is usually
orders of magnitude weaker than electromagnetic radiation.

Thus any object that is modifying its shape is a source of gravitional
waves. All orbiting systems therefore are sources of
Gravitational Wave ()s. However as said just above, only very important
`changes in shape' have a chance to be detectable. These phenomena are
what we will explore next.

\hypertarget{sec-compactbinaries}{%
\subsection{Compact binaries}\label{sec-compactbinaries}}

How could one construct a very powerful source of {GW}s? One could take
two very massive objects, (such that
\(\texttip{\texttip{T}{Stress-Energy tensor, or energy-momentum tensor}_{00}}{covariant Stress-Energy tensor component}\)
is large) and make them orbit each other. At that point one could hope
to detect this orbit if the objects are massive enough and orbiting
close enough for the `change in shape' of the system to sizeable. For
these very massive objects to be close enough for a small orbit, they
have to be very compact. Assuming that is the case, a funny thing
happens, as these objects orbit each other, they emit {GW}s, and in
doing so they lose energy\sidenote{\footnotesize of course this happens in every
  orbiting system just on a time scale that is neglible. Only systems
  which are massive enough to produce large amounts of radiation
  actually lose enough energy for it to matter}. Thus they slow down,
and their orbit shrinks. This continues until the orbit is so small that
the objects merge into a single object. Of course this is a very
important change of shape and thus we have a constructed (if not all on
purpose) a very powerful source of {GW}s. Such objects are called
compact binaries.

\hypertarget{binaries}{%
\subsection{\texorpdfstring{{BH}
Binaries}{BH Binaries}}\label{binaries}}

Taking the process described above to the limit one could imagine asking
for the most dense objects possible to orbit each other. In {GR} this
object is called a black hole ({BH}). It is a possible solution to the
full fat Einstein Field Equations () (\ref{eq-EFE}), where we consider a
static and isotropic universe, with point like mass at its center. Then
the solutions to the equations \ref{eq-EFE} have a unique form (Birkhoff
and Langer
(1923)\marginpar{\begin{footnotesize}\leavevmode\vadjust pre{\protect\hypertarget{ref-birkhoffRelativityModernPhysics1923}{}}%
Birkhoff, George David, and Rudolph Ernest Langer. 1923.
\emph{Relativity and Modern Physics}. {Cambridge}: {Harvard University
Press; {[}etc., etc.{]}}.\vspace{2mm}\par\end{footnotesize}}),
called the Schwarzschild solution. The solution is given by:

\hypertarget{binaries-1}{%
\subsection{\texorpdfstring{{NS}
Binaries}{NS Binaries}}\label{binaries-1}}

We can also not go as far as having our compact object be a {BH}, but
can also consider {NS}s. These object are notheless extremely dense (not
infinitely so), and are essentially like big atomic nuclei. Not much is
known about them and in fact {GW} astronomy is could be

\hypertarget{exotic-sources}{%
\subsection{Exotic Sources}\label{exotic-sources}}

Cosmic strings, supernovae, and other exotic sources of {GW}s are also
possible.

\bookmarksetup{startatroot}

\hypertarget{gravitational-wave-detection}{%
\chapter{Gravitational Wave
Detection}\label{gravitational-wave-detection}}

Gravitiional waves were first theorised by Einstein, accompanying his
theory of general relativity. However it was not even clear whether this
was an artefact of the theory or something real. Of course as we have
seen, {GW}s are predicted to be very weak, and thus there initiall was
little hope of ever detecting them. As evidence mounted that {GR} was
indeed a very good theory of gravity, people started to investigate more
seriously whether one could detect {GW}s. The first detectors were
resonant antennas,

thowever these have to date, not been sucessful in detecting {GW}s. The
more modern laser based detectors were the final piece of the puzzle.
With extremely high precision (the modern measurement accuracy at {LIGO}
is equivalent to measuring the distance to alpha centauri to the width
of a human hair), these detectors are able to detect {GW}s. The first
detection was made in 2015, marking the beginning of a new era in
astronomy, and maybe physics.

\hypertarget{laser-interferometers}{%
\section{Laser interferometers}\label{laser-interferometers}}

\hypertarget{sec-ligo}{%
\subsection{\texorpdfstring{{LIGO}}{LIGO}}\label{sec-ligo}}

\hypertarget{section}{%
\subsection{Laser Interferometer Space Antenna ()}\label{section}}

\hypertarget{matched-filtering}{%
\section{matched filtering}\label{matched-filtering}}

The signals measured by the laser interferometers such as {LIGO}, are a
measurement of strain over time. This waveform essentially measures the
deformation of spacetime when a {GW} passes through a detector. While
some {GW} signals coming out of laser interferometers, are clearly
identified as such, mainly due to their extraordinary power output,
smaller systems and weaker signals are harder to identify. In fact most
of the detections have happened under the noise floor of the detectors.
How is this possible? Matched filtering and waveform generation. In this
chapter we will explain the matched filtering approach and then
introduce the methods that are used to generate the waveforms to be
matched against. Waveforms for spiraling binaries are the principle tool
for comparing data from Gravitational wave detectors to the theory
surrounding them. They are the bridge between theory and experiment.

As we have seen, the matched filtering approach is contingent on having
a waveform that corresponds to the physical process that is emitting the
signal. Usually these processes can be parametrized by a set of
parameters, such as in the case of a compact binary, the mass ratio of
the binary, the orbit eccentricity, the possible spin etc. The filter
waveform must then of course depend on these parameters, be it
heuristically or physically (i.e., from first principles).

\hypertarget{pulsar-timing-arrays}{%
\section{Pulsar Timing Arrays}\label{pulsar-timing-arrays}}

\bookmarksetup{startatroot}

\hypertarget{waveform-generation}{%
\chapter{Waveform Generation}\label{waveform-generation}}

The matched filtering approach introduced in the previous section is a
powerful tool for detecting signals in noise. Additionally, if the
matching signal has some physical content, that same content can be
expected to describe the emitting object with resonable error. Ideally
the matching signal is entirely constructed from first priciples, thus
when a match is obtained the physical inputs to the model

The matched filtering approach introduced in the previous section
motivates tools for generating waveforms. These can be varied in
physical content, but also in practicality. In all cases they need to be
able to cover the set of parameters (the parameter space) that one is
interested in exploring. Clearly if the waveform has been generated
purely heuristically, then the physical content of the detection is
close to zero. One can only say one has detected a phenomenon that
produces this waveform, but not much more. However, if the input
parameters are physically meaningful, like the mass ratio of a binary
compact inspiraling system, then the waveform constructed purely from
this input, if matched to the signal, tells us that a compact binary
inspiraling system with this mass ratio has been detected. However, the
waveform generation must be able to cover the whole parameter space (in
this case all feasible mass ratios) to be useful, as one does \emph{not}
know the parameters of the object one is looking for. Ideally this
waveform generation can be done in a way that is computationally
efficient, as the number of parameters to explore can be large. In the
end, currently, waveform generation is done in a hybrid way, where some
waveforms at some parameter space points are generated from first
principles, and the rest of the parameter space waveforms are
interpolated.

In this chapter we will look at the physically motivated waveform
generation tools currently being developed in research. More
specifically we are going to explore techniques for compact binary
waveforms, as these are currently the objects to which {LIGO} and others
are most sensitive to (see Section~\ref{sec-ligo}) and by far the most
common emitters of high intensity {GW}s. As discussed in
Section~\ref{sec-compactbinaries}, orbiting and {GW} emitting compact
objects will necessarily inspiral and merge at somepoint.

For such inspiraling binaries, the waveform has three distinct parts,
corresponding to distinct phases of a binary merger: a first inspiral
phase, a merger phase where a remnant compact body is produced as a
result of the coalescence of the two objects, and a postmerger, or
ringdown phase where the remnant still emits gravitational radiation
while settling to its new stable configuration. Each phase has specific
characteristics in the dynamics of the objects and thus correspondingly
different frameworks are aimed at specific portions of the waveforms.
Even within one phase, different regimes and thus frameworks exist,
additionally dependent to the type/initial state of the two orbiting
bodies.

\bookmarksetup{startatroot}

\hypertarget{inspiral-phase}{%
\chapter{Inspiral phase}\label{inspiral-phase}}

\hypertarget{section-1}{%
\section{\texorpdfstring{{PN}}{PN}}\label{section-1}}

\hypertarget{section-2}{%
\section{\texorpdfstring{{PM}}{PM}}\label{section-2}}

\hypertarget{section-3}{%
\section{Gravitational Self-Force ()}\label{section-3}}

\hypertarget{section-4}{%
\section{\texorpdfstring{{EOB}}{EOB}}\label{section-4}}

\bookmarksetup{startatroot}

\hypertarget{merger-and-ringdown-phase}{%
\chapter{Merger and Ringdown phase}\label{merger-and-ringdown-phase}}

\hypertarget{section-5}{%
\section{\texorpdfstring{{EOB}}{EOB}}\label{section-5}}

\bookmarksetup{startatroot}

\hypertarget{scattering-amplitudes-and-gravitional-waves}{%
\chapter{Scattering amplitudes and Gravitional
waves}\label{scattering-amplitudes-and-gravitional-waves}}

Throughout we use relativistically natural units, i.e.~we do \emph{not}
set \(\hbar=1\). In dimensional analysis we can therefore see that
\(c=1\) means that \([L][T]^{-1}=1 \implies [L]=[T]\),
\[E=mc^2\implies[E]=[m]=[M]\] and
\[E=\hbar \omega\implies [M]=[\hbar][T]^{-1}\implies [\hbar]=[T][M]\].
Thus momentum \(p\) is in units of \([p]=[M]\) mass and Wavenumber
\([\bar{p}]=[\frac{p}{\hbar}]=[T]^{-1}\) is in units of inverse time.

The setup of the Kosower Maybee and O'Connell () framework (Kosower,
Maybee, and O'Connell
(2019)\marginpar{\begin{footnotesize}\leavevmode\vadjust pre{\protect\hypertarget{ref-kosowerAmplitudesObservablesClassical2019}{}}%
Kosower, David A., Ben Maybee, and Donal O'Connell. 2019. {``Amplitudes,
Observables, and Classical Scattering.''} \emph{Journal of High Energy
Physics} 2019 (2): 137. \url{https://doi.org/10.1007/JHEP02(2019)137}.\vspace{2mm}\par\end{footnotesize}})
is very general, and is aimed at taking the classical limit of a
scattering event in an unspecified theory. We will later on apply it to
\arc{sqed} and gravity. Imagine we want to scatter two particles into
each other, obtaining two particles out. In {QFT} the framework that
formalizes scattering of definite states is called the
Lehmann Symanzik and Zimmermann ()

in state into an at least 2 particle out state. This language

Thus the in state is given by

\begin{equation}\protect\hypertarget{eq-instate}{}{\oldket[in]{\psi}=\int \,\mathtip{\mathrm{d}\Phi_{2}(p_1,p_2)}{\text{On shell integration measure: } \mathrm{d}\Phi(p_i)=\,\mathtip{\tilde{\mathrm{d}}^{4}p_i}{\tilde{\mathrm{d}}^{4}p_i=\frac{\,\mathrm{d}^{4}p_i\,}{(2\pi)^{4}}}\,\mathtip{\tilde{\delta}^{(+)}\left(p_i^2-m_i^2\right)}{\text{Normalized positive energy delta function: }\tilde{\delta}^{(+)}\left(p_i^2-m_i^2\right)=\texttip{\delta^{(+)}\left(p_i^2-m_i^2\right)}{Dirac delta function} 2\pi= 2 \pi \texttip{\delta^{()}\left(p_i^2 -m_i^2\right)}{Dirac delta function}\texttip{\Theta^{()}\left(p_i^0\right)}{Heaviside step function}}}\, \phi_1(p_1) \phi_2(p_2) \oldexp{\oldi b_\mu p^\mu_1 /\hbar}\oldket{p_1,p_2}_\text{in}}\label{eq-instate}\end{equation}

The \(\oldexp{\oldi b_\mu p^\mu_1 /\hbar}\) factor encodes the fact that
we have translated the wavepacket of particle \(1\) relative to particle
\(2\) by the impact parameter \(b\) .{[}\^{}2{]} We take it to be
perpendicular to the initial momenta \(p_1,p_2\).

The {KMOC} framework concerns itself with the change of an observable
during a scattering event. For such an observable \({O}\), its change
can be simply obtained by evaluating the difference of the expectation
value of the corresponding Hermitian operator, \(\mathbb{O}\), between
in and out states \[
\Delta O=\mathtip{\langle\text{out} \vert  \mathtip{\mathbb{O}}{\text{Operator: }\mathbb{O}} \vert \text{out} \rangle}{\text{Expectation value: }\langle\text{out} \vert  \mathtip{\mathbb{O}}{\text{Operator: }\mathbb{O}} \vert \text{out} \rangle}-\mathtip{\langle\text{in} \vert  \mathtip{\mathbb{O}}{\text{Operator: }\mathbb{O}} \vert \text{in} \rangle}{\text{Expectation value: }\langle\text{in} \vert  \mathtip{\mathbb{O}}{\text{Operator: }\mathbb{O}} \vert \text{in} \rangle}
\] In quantum mechanics, the out states are related to the in states by
the time evolution operator, i.e.~the S-matrix:
\(\oldket{\text{out}}=\oldS\oldket{\text{in}}\) and we can write
\begin{equation}\protect\hypertarget{eq-DeltaO}{}{\begin{aligned}
\Delta O&=\mathtip{\langle\text{in} \vert  \oldS^\dagger\mathtip{\mathbb{O}}{\text{Operator: }\mathbb{O}}\oldS \vert \text{in} \rangle}{\text{Expectation value: }\langle\text{in} \vert  \oldS^\dagger\mathtip{\mathbb{O}}{\text{Operator: }\mathbb{O}}\oldS \vert \text{in} \rangle}-\mathtip{\langle\text{in} \vert  \mathtip{\mathbb{O}}{\text{Operator: }\mathbb{O}} \vert \text{in} \rangle}{\text{Expectation value: }\langle\text{in} \vert  \mathtip{\mathbb{O}}{\text{Operator: }\mathbb{O}} \vert \text{in} \rangle}\\
&\stackrel{\oldS^\dagger \oldS=1}{=} \mathtip{\langle\text{in} \vert  \oldS^\dagger \mathtip{[\mathtip{\mathbb{O}}{\text{Operator: }\mathbb{O}},\oldS]}{\text{Commutator: }[\mathtip{\mathbb{O}}{\text{Operator: }\mathbb{O}},\oldS]} \vert \text{in} \rangle}{\text{Expectation value: }\langle\text{in} \vert  \oldS^\dagger \mathtip{[\mathtip{\mathbb{O}}{\text{Operator: }\mathbb{O}},\oldS]}{\text{Commutator: }[\mathtip{\mathbb{O}}{\text{Operator: }\mathbb{O}},\oldS]} \vert \text{in} \rangle}\\
&\stackrel{\oldS=1+\oldi\oldT}{=} \mathtip{\langle\text{in} \vert  \mathtip{[\mathtip{\mathbb{O}}{\text{Operator: }\mathbb{O}},1+\oldi\oldT]}{\text{Commutator: }[\mathtip{\mathbb{O}}{\text{Operator: }\mathbb{O}},1+\oldi\oldT]} \vert \text{in} \rangle}{\text{Expectation value: }\langle\text{in} \vert  \mathtip{[\mathtip{\mathbb{O}}{\text{Operator: }\mathbb{O}},1+\oldi\oldT]}{\text{Commutator: }[\mathtip{\mathbb{O}}{\text{Operator: }\mathbb{O}},1+\oldi\oldT]} \vert \text{in} \rangle}-\mathtip{\langle\text{in} \vert  \oldi\oldT^\dagger \mathtip{[\mathtip{\mathbb{O}}{\text{Operator: }\mathbb{O}},1+\oldi\oldT]}{\text{Commutator: }[\mathtip{\mathbb{O}}{\text{Operator: }\mathbb{O}},1+\oldi\oldT]} \vert \text{in} \rangle}{\text{Expectation value: }\langle\text{in} \vert  \oldi\oldT^\dagger \mathtip{[\mathtip{\mathbb{O}}{\text{Operator: }\mathbb{O}},1+\oldi\oldT]}{\text{Commutator: }[\mathtip{\mathbb{O}}{\text{Operator: }\mathbb{O}},1+\oldi\oldT]} \vert \text{in} \rangle}\\
&=\mathtip{\langle\text{in} \vert  \oldi\mathtip{[\mathtip{\mathbb{O}}{\text{Operator: }\mathbb{O}},\oldT]}{\text{Commutator: }[\mathtip{\mathbb{O}}{\text{Operator: }\mathbb{O}},\oldT]} \vert \text{in} \rangle}{\text{Expectation value: }\langle\text{in} \vert  \oldi\mathtip{[\mathtip{\mathbb{O}}{\text{Operator: }\mathbb{O}},\oldT]}{\text{Commutator: }[\mathtip{\mathbb{O}}{\text{Operator: }\mathbb{O}},\oldT]} \vert \text{in} \rangle}+\mathtip{\langle\text{in} \vert   \oldT^\dagger \mathtip{[\mathtip{\mathbb{O}}{\text{Operator: }\mathbb{O}},\oldT]}{\text{Commutator: }[\mathtip{\mathbb{O}}{\text{Operator: }\mathbb{O}},\oldT]} \vert \text{in} \rangle}{\text{Expectation value: }\langle\text{in} \vert   \oldT^\dagger \mathtip{[\mathtip{\mathbb{O}}{\text{Operator: }\mathbb{O}},\oldT]}{\text{Commutator: }[\mathtip{\mathbb{O}}{\text{Operator: }\mathbb{O}},\oldT]} \vert \text{in} \rangle}\\
&=\Delta O_\text{v}+\Delta O_\text{r}
\end{aligned}}\label{eq-DeltaO}\end{equation}

If we put in the definition of our in state (\ref{eq-instate}) we have:

\[\Delta O = \int \,\mathtip{\mathrm{d}\Phi_{4}(p_1,p_2,p_1',p_2')}{\text{On shell integration measure: } \mathrm{d}\Phi(p_i)=\,\mathtip{\tilde{\mathrm{d}}^{4}p_i}{\tilde{\mathrm{d}}^{4}p_i=\frac{\,\mathrm{d}^{4}p_i\,}{(2\pi)^{4}}}\,\mathtip{\tilde{\delta}^{(+)}\left(p_i^2-m_i^2\right)}{\text{Normalized positive energy delta function: }\tilde{\delta}^{(+)}\left(p_i^2-m_i^2\right)=\texttip{\delta^{(+)}\left(p_i^2-m_i^2\right)}{Dirac delta function} 2\pi= 2 \pi \texttip{\delta^{()}\left(p_i^2 -m_i^2\right)}{Dirac delta function}\texttip{\Theta^{()}\left(p_i^0\right)}{Heaviside step function}}}\, \phi_1(p_1)\phi_2(p_2)\phi_1^\dagger(p_1')\phi_2^\dagger(p_2')\,\oldexp{\oldi b_\mu\frac{p_1^\mu-p_1'^\mu}{\hbar}}[\mathcal{I_\text{v}}(O)-\mathtip{\mathcal{I}_{\text{r}}(\mathtip{\mathcal{O}}{\text{Observable: }\mathcal{O}})}{\text{Real integrand: }\mathcal{I}_{\text{r}}(\mathtip{\mathcal{O}}{\text{Observable: }\mathcal{O}})} ]\]
Where \[\begin{aligned}
\mathtip{\mathcal{I}_{\text{v}}(\mathtip{\mathcal{O}}{\text{Observable: }\mathcal{O}})}{\text{Virtual integrand: }\mathcal{I}_{\text{v}}(\mathtip{\mathcal{O}}{\text{Observable: }\mathcal{O}})}&=\mathtip{\langle p_1'p_2' \vert \oldi\mathtip{[\mathtip{\mathbb{O}}{\text{Operator: }\mathbb{O}},\oldT]}{\text{Commutator: }[\mathtip{\mathbb{O}}{\text{Operator: }\mathbb{O}},\oldT]} \vert p_1p_2 \rangle}{\text{Matrix element: }\langle p_1'p_2' \vert \oldi\mathtip{[\mathtip{\mathbb{O}}{\text{Operator: }\mathbb{O}},\oldT]}{\text{Commutator: }[\mathtip{\mathbb{O}}{\text{Operator: }\mathbb{O}},\oldT]} \vert p_1p_2 \rangle}\\
\mathtip{\mathcal{I}_{\text{r}}(\mathtip{\mathcal{O}}{\text{Observable: }\mathcal{O}})}{\text{Real integrand: }\mathcal{I}_{\text{r}}(\mathtip{\mathcal{O}}{\text{Observable: }\mathcal{O}})}&=\mathtip{\langle p_1'p_2' \vert \oldT^\dagger\mathtip{[\mathtip{\mathbb{O}}{\text{Operator: }\mathbb{O}},\oldT]}{\text{Commutator: }[\mathtip{\mathbb{O}}{\text{Operator: }\mathbb{O}},\oldT]} \vert p_1p_2 \rangle}{\text{Matrix element: }\langle p_1'p_2' \vert \oldT^\dagger\mathtip{[\mathtip{\mathbb{O}}{\text{Operator: }\mathbb{O}},\oldT]}{\text{Commutator: }[\mathtip{\mathbb{O}}{\text{Operator: }\mathbb{O}},\oldT]} \vert p_1p_2 \rangle}
\end{aligned}\]

\begin{quote}
NB: the notation is slightly different in the Bern et al.
(2021)\marginpar{\begin{footnotesize}\leavevmode\vadjust pre{\protect\hypertarget{ref-bernScalarQEDToy2021}{}}%
Bern, Zvi, Juan Pablo Gatica, Enrico Herrmann, Andres Luna, and Mao
Zeng. 2021. {``Scalar {QED} as a Toy Model for Higher-Order Effects in
Classical Gravitational Scattering.''} \emph{arXiv:2112.12243 {[}Gr-Qc,
Physics:hep-Th{]}}, December. \url{https://arxiv.org/abs/2112.12243}.\vspace{2mm}\par\end{footnotesize}}
paper
\end{quote}

Let us first look at the virtual integrand \(\mathcal{I}_\text{v}\):

\[\begin{aligned}
\mathtip{\mathcal{I}_{\text{v}}(\mathtip{\mathcal{O}}{\text{Observable: }\mathcal{O}})}{\text{Virtual integrand: }\mathcal{I}_{\text{v}}(\mathtip{\mathcal{O}}{\text{Observable: }\mathcal{O}})}&=\mathtip{\langle p_1'p_2' \vert \oldi\mathtip{[\mathtip{\mathbb{O}}{\text{Operator: }\mathbb{O}},\oldT]}{\text{Commutator: }[\mathtip{\mathbb{O}}{\text{Operator: }\mathbb{O}},\oldT]} \vert p_1p_2 \rangle}{\text{Matrix element: }\langle p_1'p_2' \vert \oldi\mathtip{[\mathtip{\mathbb{O}}{\text{Operator: }\mathbb{O}},\oldT]}{\text{Commutator: }[\mathtip{\mathbb{O}}{\text{Operator: }\mathbb{O}},\oldT]} \vert p_1p_2 \rangle}  \\
      &=\mathtip{\langle p_1'p_2' \vert \oldi\mathtip{\mathbb{O}}{\text{Operator: }\mathbb{O}}\oldT \vert p_1p_2 \rangle}{\text{Matrix element: }\langle p_1'p_2' \vert \oldi\mathtip{\mathbb{O}}{\text{Operator: }\mathbb{O}}\oldT \vert p_1p_2 \rangle}   - \mathtip{\langle p_1'p_2' \vert \oldi\oldT\mathtip{\mathbb{O}}{\text{Operator: }\mathbb{O}}  \vert p_1p_2 \rangle}{\text{Matrix element: }\langle p_1'p_2' \vert \oldi\oldT\mathtip{\mathbb{O}}{\text{Operator: }\mathbb{O}}  \vert p_1p_2 \rangle}\\
      &=\oldi\mathtip{\mathcal{O}}{\text{Observable: }\mathcal{O}}_{\text{in}'} \mathtip{\langle p_1'p_2' \vert \oldT \vert p_1p_2 \rangle}{\text{Matrix element: }\langle p_1'p_2' \vert \oldT \vert p_1p_2 \rangle}- \oldi\mathtip{\mathcal{O}}{\text{Observable: }\mathcal{O}}_{\text{in}}\mathtip{\langle p_1'p_2' \vert \oldT \vert p_1p_2 \rangle}{\text{Matrix element: }\langle p_1'p_2' \vert \oldT \vert p_1p_2 \rangle}\\
      &=\oldi\mathtip{\underset{p-p'}{\Delta} \!\mathtip{\mathcal{O}}{\text{Observable: }\mathcal{O}}\,}{\text{Observable change: }\underset{p-p'}{\Delta} \!\mathtip{\mathcal{O}}{\text{Observable: }\mathcal{O}}\,} \, \mathtip{\langle p_1'p_2' \vert \oldT \vert p_1p_2 \rangle}{\text{Matrix element: }\langle p_1'p_2' \vert \oldT \vert p_1p_2 \rangle}\\
      &=\oldi\mathtip{\underset{p-p'}{\Delta} \!\mathtip{\mathcal{O}}{\text{Observable: }\mathcal{O}}\,}{\text{Observable change: }\underset{p-p'}{\Delta} \!\mathtip{\mathcal{O}}{\text{Observable: }\mathcal{O}}\,}\,\mathtip{\tilde{\delta}^{(4)}\left(p_1+p_2-p_1'-p_2'\right)}{\text 
{Normalized Dirac delta function: } \tilde{\delta}^{(4)}\left(p_1+p_2-p_1'-p_2'\right) = (2\pi)^{4}\delta^{4}\left(p_1+p_2-p_1'-p_2'\right)} \mathtip{\mathcal{A}(p_1,p_2\to p_1',p_2')}{\text{Amplitude: }\mathcal{A}(p_1,p_2\to p_1',p_2')}
\end{aligned}\]

Note that the amplitude is from in states to in states! Now for the real
integrand \(\mathcal{I}_\text{r}\) we insert a complete set of states
:{[}\^{}3{]}

\$\$

\begin{aligned}
\mathtip{\mathcal{I}_{\text{r}}(\mathtip{\mathcal{O}}{\text{Observable: }\mathcal{O}})}{\text{Real integrand: }\mathcal{I}_{\text{r}}(\mathtip{\mathcal{O}}{\text{Observable: }\mathcal{O}})}&=\mathtip{\langle p_1'p_2' \vert \oldT^\dagger\mathtip{[\mathtip{\mathbb{O}}{\text{Operator: }\mathbb{O}},\oldT]}{\text{Commutator: }[\mathtip{\mathbb{O}}{\text{Operator: }\mathbb{O}},\oldT]} \vert p_1p_2 \rangle}{\text{Matrix element: }\langle p_1'p_2' \vert \oldT^\dagger\mathtip{[\mathtip{\mathbb{O}}{\text{Operator: }\mathbb{O}},\oldT]}{\text{Commutator: }[\mathtip{\mathbb{O}}{\text{Operator: }\mathbb{O}},\oldT]} \vert p_1p_2 \rangle}\\
&=\sum\limits_X \int \,\mathtip{\mathrm{d}\Phi_{2+\vert X \vert}(r_1,r_2,X)}{\text{On shell integration measure: } \mathrm{d}\Phi(p_i)=\,\mathtip{\tilde{\mathrm{d}}^{4}p_i}{\tilde{\mathrm{d}}^{4}p_i=\frac{\,\mathrm{d}^{4}p_i\,}{(2\pi)^{4}}}\,\mathtip{\tilde{\delta}^{(+)}\left(p_i^2-m_i^2\right)}{\text{Normalized positive energy delta function: }\tilde{\delta}^{(+)}\left(p_i^2-m_i^2\right)=\texttip{\delta^{(+)}\left(p_i^2-m_i^2\right)}{Dirac delta function} 2\pi= 2 \pi \texttip{\delta^{()}\left(p_i^2 -m_i^2\right)}{Dirac delta function}\texttip{\Theta^{()}\left(p_i^0\right)}{Heaviside step function}}}\,\mathtip{\langle p_1'p_2' \vert \oldT^\dagger \vert r_1 r_2 X \rangle}{\text{Matrix element: }\langle p_1'p_2' \vert \oldT^\dagger \vert r_1 r_2 X \rangle}\mathtip{\langle r_1 r_2 X \vert \mathtip{[\mathtip{\mathbb{O}}{\text{Operator: }\mathbb{O}},\oldT]}{\text{Commutator: }[\mathtip{\mathbb{O}}{\text{Operator: }\mathbb{O}},\oldT]} \vert p_1p_2 \rangle}{\text{Matrix element: }\langle r_1 r_2 X \vert \mathtip{[\mathtip{\mathbb{O}}{\text{Operator: }\mathbb{O}},\oldT]}{\text{Commutator: }[\mathtip{\mathbb{O}}{\text{Operator: }\mathbb{O}},\oldT]} \vert p_1p_2 \rangle}\\
&=\sum\limits_X \int \,\mathtip{\mathrm{d}\Phi_{2+\vert X \vert}(r_1,r_2,X)}{\text{On shell integration measure: } \mathrm{d}\Phi(p_i)=\,\mathtip{\tilde{\mathrm{d}}^{4}p_i}{\tilde{\mathrm{d}}^{4}p_i=\frac{\,\mathrm{d}^{4}p_i\,}{(2\pi)^{4}}}\,\mathtip{\tilde{\delta}^{(+)}\left(p_i^2-m_i^2\right)}{\text{Normalized positive energy delta function: }\tilde{\delta}^{(+)}\left(p_i^2-m_i^2\right)=\texttip{\delta^{(+)}\left(p_i^2-m_i^2\right)}{Dirac delta function} 2\pi= 2 \pi \texttip{\delta^{()}\left(p_i^2 -m_i^2\right)}{Dirac delta function}\texttip{\Theta^{()}\left(p_i^0\right)}{Heaviside step function}}}\,  \hat{\delta}^{(4)}\left(p_{1}+p_{2}-r_{1}-r_{2}-r_{X}\right)\mathcal{A}\left(p_{1}, p_{2} \rightarrow r_{1}, r_{2}, r_{X}\right) \\
&\qquad \underset{rX-p}{\Delta}\!\! O \,\hat{\delta}^{(4)}\left(p_{1}^{\prime}+p_{2}^{\prime}-r_{1}-r_{2}-r_{X}\right)   \mathcal{A}^{*}\left(p_{1}^{\prime}, p_{2}^{\prime} \rightarrow r_{1}, r_{2}, r_{X}\right)

\end{aligned}

\$\$

For both integrands we can preform some variable changes and eliminate
certain delta functions. We introduce momentum shifts \(q_i=p'_i-p_i\)
and then integrate over \(q_2\) finally relabelling
\(q_1 \to q\).{[}\^{}4{]} Thus we have

\$\$

\begin{aligned}
\Delta O_\text{v}=\int \,\mathtip{\mathrm{d}\Phi_{2}(p_1,p_2)}{\text{On shell integration measure: } \mathrm{d}\Phi(p_i)=\,\mathtip{\tilde{\mathrm{d}}^{4}p_i}{\tilde{\mathrm{d}}^{4}p_i=\frac{\,\mathrm{d}^{4}p_i\,}{(2\pi)^{4}}}\,\mathtip{\tilde{\delta}^{(+)}\left(p_i^2-m_i^2\right)}{\text{Normalized positive energy delta function: }\tilde{\delta}^{(+)}\left(p_i^2-m_i^2\right)=\texttip{\delta^{(+)}\left(p_i^2-m_i^2\right)}{Dirac delta function} 2\pi= 2 \pi \texttip{\delta^{()}\left(p_i^2 -m_i^2\right)}{Dirac delta function}\texttip{\Theta^{()}\left(p_i^0\right)}{Heaviside step function}}}\, \dh[4]q\, &\hat{\delta}(2 p_1 \cdot q+q^2) \Theta(p^0_1+q^0)\, \hat{\delta}(2 p_2 \cdot q-q^2) \Theta(p^0_2-q^0)\\
&\times\phi_1(p_1)\phi_2(p_2)\phi_1^\dagger(p_1+q)\phi_2^\dagger(p_2-q)\,\oldexp{-\frac{\oldi}{\hbar} b_\mu q^\mu}\\
&\times \oldi\underset{q}{\Delta} O\, \mathcal{A}(p_1,p_2 \to p_1+q,p_2-q)

\end{aligned}

\$\$\{\#eq-DeltaOv\}

\bookmarksetup{startatroot}

\hypertarget{summary}{%
\chapter{Summary}\label{summary}}

In summary, this book has no content whatsoever.

\bookmarksetup{startatroot}

\hypertarget{references}{%
\chapter*{References}\label{references}}
\addcontentsline{toc}{chapter}{References}




\end{document}
