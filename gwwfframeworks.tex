% Options for packages loaded elsewhere
\PassOptionsToPackage{unicode}{hyperref}
\PassOptionsToPackage{hyphens}{url}
\PassOptionsToPackage{dvipsnames,svgnames,x11names}{xcolor}
%
\documentclass[
  letterpaper,
  DIV=11,
  numbers=noendperiod]{scrartcl}

\usepackage{amsmath,amssymb}
\usepackage{lmodern}
\usepackage{iftex}
\ifPDFTeX
  \usepackage[T1]{fontenc}
  \usepackage[utf8]{inputenc}
  \usepackage{textcomp} % provide euro and other symbols
\else % if luatex or xetex
  \usepackage{unicode-math}
  \defaultfontfeatures{Scale=MatchLowercase}
  \defaultfontfeatures[\rmfamily]{Ligatures=TeX,Scale=1}
\fi
% Use upquote if available, for straight quotes in verbatim environments
\IfFileExists{upquote.sty}{\usepackage{upquote}}{}
\IfFileExists{microtype.sty}{% use microtype if available
  \usepackage[]{microtype}
  \UseMicrotypeSet[protrusion]{basicmath} % disable protrusion for tt fonts
}{}
\makeatletter
\@ifundefined{KOMAClassName}{% if non-KOMA class
  \IfFileExists{parskip.sty}{%
    \usepackage{parskip}
  }{% else
    \setlength{\parindent}{0pt}
    \setlength{\parskip}{6pt plus 2pt minus 1pt}}
}{% if KOMA class
  \KOMAoptions{parskip=half}}
\makeatother
\usepackage{xcolor}
\setlength{\emergencystretch}{3em} % prevent overfull lines
\setcounter{secnumdepth}{-\maxdimen} % remove section numbering
% Make \paragraph and \subparagraph free-standing
\ifx\paragraph\undefined\else
  \let\oldparagraph\paragraph
  \renewcommand{\paragraph}[1]{\oldparagraph{#1}\mbox{}}
\fi
\ifx\subparagraph\undefined\else
  \let\oldsubparagraph\subparagraph
  \renewcommand{\subparagraph}[1]{\oldsubparagraph{#1}\mbox{}}
\fi


\providecommand{\tightlist}{%
  \setlength{\itemsep}{0pt}\setlength{\parskip}{0pt}}\usepackage{longtable,booktabs,array}
\usepackage{calc} % for calculating minipage widths
% Correct order of tables after \paragraph or \subparagraph
\usepackage{etoolbox}
\makeatletter
\patchcmd\longtable{\par}{\if@noskipsec\mbox{}\fi\par}{}{}
\makeatother
% Allow footnotes in longtable head/foot
\IfFileExists{footnotehyper.sty}{\usepackage{footnotehyper}}{\usepackage{footnote}}
\makesavenoteenv{longtable}
\usepackage{graphicx}
\makeatletter
\def\maxwidth{\ifdim\Gin@nat@width>\linewidth\linewidth\else\Gin@nat@width\fi}
\def\maxheight{\ifdim\Gin@nat@height>\textheight\textheight\else\Gin@nat@height\fi}
\makeatother
% Scale images if necessary, so that they will not overflow the page
% margins by default, and it is still possible to overwrite the defaults
% using explicit options in \includegraphics[width, height, ...]{}
\setkeys{Gin}{width=\maxwidth,height=\maxheight,keepaspectratio}
% Set default figure placement to htbp
\makeatletter
\def\fps@figure{htbp}
\makeatother

\KOMAoption{captions}{tableheading}
\makeatletter
\makeatother
\makeatletter
\makeatother
\makeatletter
\@ifpackageloaded{caption}{}{\usepackage{caption}}
\AtBeginDocument{%
\ifdefined\contentsname
  \renewcommand*\contentsname{Table of contents}
\else
  \newcommand\contentsname{Table of contents}
\fi
\ifdefined\listfigurename
  \renewcommand*\listfigurename{List of Figures}
\else
  \newcommand\listfigurename{List of Figures}
\fi
\ifdefined\listtablename
  \renewcommand*\listtablename{List of Tables}
\else
  \newcommand\listtablename{List of Tables}
\fi
\ifdefined\figurename
  \renewcommand*\figurename{Figure}
\else
  \newcommand\figurename{Figure}
\fi
\ifdefined\tablename
  \renewcommand*\tablename{Table}
\else
  \newcommand\tablename{Table}
\fi
}
\@ifpackageloaded{float}{}{\usepackage{float}}
\floatstyle{ruled}
\@ifundefined{c@chapter}{\newfloat{codelisting}{h}{lop}}{\newfloat{codelisting}{h}{lop}[chapter]}
\floatname{codelisting}{Listing}
\newcommand*\listoflistings{\listof{codelisting}{List of Listings}}
\makeatother
\makeatletter
\@ifpackageloaded{caption}{}{\usepackage{caption}}
\@ifpackageloaded{subcaption}{}{\usepackage{subcaption}}
\makeatother
\makeatletter
\@ifpackageloaded{tcolorbox}{}{\usepackage[many]{tcolorbox}}
\makeatother
\makeatletter
\@ifundefined{shadecolor}{\definecolor{shadecolor}{rgb}{.97, .97, .97}}
\makeatother
\makeatletter
\makeatother
\ifLuaTeX
  \usepackage{selnolig}  % disable illegal ligatures
\fi
\IfFileExists{bookmark.sty}{\usepackage{bookmark}}{\usepackage{hyperref}}
\IfFileExists{xurl.sty}{\usepackage{xurl}}{} % add URL line breaks if available
\urlstyle{same} % disable monospaced font for URLs
\hypersetup{
  colorlinks=true,
  linkcolor={blue},
  filecolor={Maroon},
  citecolor={Blue},
  urlcolor={Blue},
  pdfcreator={LaTeX via pandoc}}

\author{}
\date{}

\begin{document}
\ifdefined\Shaded\renewenvironment{Shaded}{\begin{tcolorbox}[sharp corners, enhanced, frame hidden, boxrule=0pt, interior hidden, breakable, borderline west={3pt}{0pt}{shadecolor}]}{\end{tcolorbox}}\fi

\hypertarget{waveform-generation}{%
\section{Waveform Generation}\label{waveform-generation}}

The matched filtering approach introduced in the previous section is a
powerful tool for detecting signals in noise. Additionally, if the
matching signal has some physical content, that same content can be
expected to describe the emitting object with resonable error. Ideally
the matching signal is entirely constructed from first priciples, thus
when a match is obtained the physical inputs to the model

The matched filtering approach introduced in the previous section
motivates tools for generating waveforms. These can be varied in
physical content, but also in practicality. In all cases they need to be
able to cover the set of parameters (the parameter space) that one is
interested in exploring. Clearly if the waveform has been generated
purely heuristically, then the physical content of the detection is
close to zero. One can only say one has detected a phenomenon that
produces this waveform, but not much more. However, if the input
parameters are physically meaningful, like the mass ratio of a binary
compact inspiraling system, then the waveform constructed purely from
this input, if matched to the signal, tells us that a compact binary
inspiraling system with this mass ratio has been detected. However, the
waveform generation must be able to cover the whole parameter space (in
this case all feasible mass ratios) to be useful, as one does \emph{not}
know the parameters of the object one is looking for. Ideally this
waveform generation can be done in a way that is computationally
efficient, as the number of parameters to explore can be large. In the
end, currently, waveform generation is done in a hybrid way, where some
waveforms at some parameter space points are generated from first
principles, and the rest of the parameter space waveforms are
interpolated.

In this chapter we will look at the physically motivated waveform
generation tools currently being developed in research. More
specifically we are going to explore techniques for compact binary
waveforms, as these are currently the objects to which \acr{ligo} and
others are most sensitive to (see \textbf{?@sec-ligo}) and by far the
most common emitters of high intensity \acr{gw}s. As discussed in
\textbf{?@sec-compactbinaries}, orbiting and \acr{gw} emitting compact
objects will necessarily inspiral and merge at somepoint.

For such inspiraling binaries, the waveform has three distinct parts,
corresponding to distinct phases of a binary merger: a first inspiral
phase, a merger phase where a remnant compact body is produced as a
result of the coalescence of the two objects, and a postmerger, or
ringdown phase where the remnant still emits gravitational radiation
while settling to its new stable configuration. Each phase has specific
characteristics in the dynamics of the objects and thus correspondingly
different frameworks are aimed at specific portions of the waveforms.
Even within one phase, different regimes and thus frameworks exist,
additionally dependent to the type/initial state of the two orbiting
bodies.

\hypertarget{inspiral-phase}{%
\section{Inspiral phase}\label{inspiral-phase}}

\hypertarget{section}{%
\subsection{\texorpdfstring{\acr{pn}}{}}\label{section}}

\hypertarget{section-1}{%
\subsection{\texorpdfstring{\acr{pm}}{}}\label{section-1}}

\hypertarget{section-2}{%
\subsection{\texorpdfstring{\acr{gsf}}{}}\label{section-2}}

\hypertarget{section-3}{%
\subsection{\texorpdfstring{\acr{eob}}{}}\label{section-3}}

\hypertarget{merger-and-ringdown-phase}{%
\section{Merger and Ringdown phase}\label{merger-and-ringdown-phase}}

\hypertarget{section-4}{%
\subsection{\texorpdfstring{\acr{eob}}{}}\label{section-4}}



\end{document}
